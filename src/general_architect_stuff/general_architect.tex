% Preamble
\documentclass[11pt]{article}

% Packages
\usepackage{amsmath}

% Document
\begin{document}

%TODO add useful stuff from here

\chapter{Introduction}

\section{Developer To Architect}

\section{Introduction to Developer To Architect}

\chapter{Performance}

\section{Module contents overview}

\section{A reference software system for discussing performance}

\section{What is performance}

\section{How do performance problems look like}

\section{Performance principles}

\section{System performance principles}

\section{Performance measurement metrics}

\section{Serial request latency}

\section{Network transfer latency}

\section{Minimising network transfer latency}

\section{Memory access latency}

\section{Minimising memory access latency}

\section{Disk access latency}

\section{Minimising disk latency}

\section{CPU processing latency}

\section{Minimising CPU processing latency}

\section{Some common latency costs}

\section{Concurrency related latency}

\section{Amdahl's law for concurrent tasks}

\section{Gunther's universal scalability law}

\section{Share resource contention}

\section{Minimising shared resource contention}

\section{Minimising lacking related contention}

\section{Pessimistic Locking}

\section{Optimistic Locking}

\section{Compare and swap mechanism}

\section{Deadlocks}

\section{Coherence related delays}

\section{Caching}

\section{System architecture for performance}

\section{Caching for performance}

\section{HTTP Caching of static data}

\section{System architecture for performance}

\section{Caching for performance}

\section{HTTP Caching of static data}

\section{Caching of dynamic data}

\section{Caching related challenges}

\section{Summary}

\section{Performance presentation slides}

\chapter{Scalability}

\section{Module contents overview}

\section{Performance vs Scalability}

\section{Vertical and Horizontal Scalability}

\section{Reverse Proxy}

\section{A reference software system for discussing scalbility}

\section{Scalability principles}

\section{Modularity for scalability}

\section{Replication}

\section{Stateful replication in web applications}

\section{Stateless replication in web applications}

\section{Stateless replication of services}

\section{Database replication}

\section{Database replication types}

\section{Need for specialised services}

\section{Specialised services - SOAP/REST}

\section{Asynchronous services}

\section{Asynchrnous processing and scalability}

\section{Caching for scalability}

\section{Verticle partitioning with micro-services}

\section{Database partitioning}

\section{Database partitioning selection}

\section{Routing with database partitioning}

\section{Methods for horizontal scalability}

\section{Dealing with large scale systems}

\section{Load balancing multiple instances}

\section{Discovery service and load balancing}

\section{Load balancer discovery}

\section{HLB vs SLB}

\section{Layer-7 load balancers}

\section{DNS as a load balancer}

\section{Global server load balancing}

\section{Global data replication}

\section{Auto scaling instances}

\section{Micro-services Architecture}

\section{Micro-services Motivation}

\section{Service Orientated Architecture}

\section{Micro-service architecture style}

\section{Transactions in micro-services}

\section{Compensating Transactions - SAGA Pattern}

\section{Microservices communication model}

\section{Event driven transactions}

\section{Extreme scalability with NoSQL and Kafka}

\section{Summary}

\chapter{Reliability}

\section{Module contents overview}

\section{Failures in large scala distributed systems}

\section{Partial system failures}

\section{Reliabilty engineering topics}

\section{Reliability}

\section{Availability}

\section{High Availability}

\section{Fault Tolerance}

\section{Designing Fault Tolerance}

\section{Fault Tolerant Design}

\section{Redundancy}

\section{Types of Redundancy}

\section{Single Point of Failure}

\section{Stateless component redundancy}

\section{Stateful component redundancy}

\section{Load balancer redundancy}

\section{Datacentre infrastructure as SPOF}

\section{Creating datacentre redundancy}

\section{Fault detection}

\section{Fault models}

\section{Health checks}

\section{External monitoring service}

\section{Internal cluster monitoring}

\section{Fault detection in a system}

\section{Recovering from failures}

\section{Stateless component recovery}

\section{Stateful failovers}

\section{Load Balancer high availability}

\section{Database recovery with hot standby}

\section{Database recovery with warm standby}

\section{Database recovery with cold backups}

\section{High availability in large scale systems}

\section{Failover best practices}

\section{System stability}

\section{Timeouts}

\section{Retries}

\section{Circuit Breaker}

\section{Fail Fast and Shed Load}

\section{Summary}

\section{Reliability Presentation Slides}

\chapter{Security}

\section{Module contents overview}

\section{Security objectives}

\section{Network security}

\section{Symmetric key encryption}

\section{Public key encryption}

\section{Secure network protocol}

\section{SSL and TLS}

\section{Hashing}

\section{Digital Signatures}

\section{Digital Certificates}

\section{Chain of Trust}

\section{TLS/SSL handshake}

\section{Secure network channel}

\section{Firewallls}

\section{Network security}

\section{Identity Mangement}

\section{Authentication and Authorisation}

\section{Authentication}

\section{Credentials transfer}

\section{Credentials verification}

\section{Stateful authentication}

\section{Single Sign-On}

\section{Access Management}

\section{Role Based Access control model}

\section{Role based access example}

\section{Authorisation}

\section{OAuth2 token grant}

\section{OAuth2 token grant - Code Flow}

\section{OAuth2 token grant - Password Flow}

\section{OAuth2 in a system}

\section{OAuth2 token types}

\section{Token storage}

\section{Securing data at rest}

\section{Securing a Software System}

\section{Common vulnerabilities}

\section{SQL Injection}

\section{Cross Site Scripting}

\section{Cross Site Resource Forgery}

\section{Summary}

\section{Security presentation slides}

\chapter{Security}

\section{Module contents overview}

\section{Security objectives}

\section{Network security}

\section{Symmetric key encryption}

\section{Public key encryption}

\section{Secure network protocol}

\section{SSL and TLS}

\section{Hashing}

\section{Digital Signatures}

\section{Digital certication}

\section{Chain of trust}

\section{TLS / SSL handshake}

\section{Secure network channel}

\section{Firewalls}

\section{Network Security}

\section{Identity management}

\section{Authentication and authorisation}

\section{Authentication}

\section{Credentials Transfer}

\section{Credentials Verification}

\section{Stateful authentication}

\section{Stateless Authentication}

\section{Single Sign-On}

\section{Access Management}

\section{Role based access control model}

\section{Role based access example}

\section{Authorisation}

\section{OAuth2 Token Grant}

\section{OAuth2 Token Grant - Password Flow}

\section{OAuth2 token types}

\section{JSON Web Tokens}

\section{Token Storage}

\section{Securing data at rest}

\section{Securing a Software System}

\section{Common Vulnerabilities}

\section{SQL Injection}

\section{Cross Site Scripting}

\section{Cross Site Resource Forgery}

\section{Summary}

\chapter{Deployment}

\section{Module contents overview}

\section{Large scale deployment challenges}

\section{Application Deployment}

\section{Infrastructure deployment}

\section{System operations}

\section{Modern deployment solutions}

\section{Application Deployment}

\section{Component Deployment}

\section{Component Deployment automation}

\section{Deployment with Virtual Machines}

\section{Isolation Through Virtual Machines}

\section{Deployment with Containers}

\section{Docker Containers}

\section{Infrastructure Deployment}

\section{Infrastructure Requirements}

\section{Provisioning and Configuration}

\section{Deployment with Containers on Cloud}

\section{Deployment with Containers with AWS cloud stack}

\section{Deployment with Kubernetes}

\section{Kubernetes Lifecycle Mangement}

\section{Kubernetes naming and addressing}

\section{Kubernetes Scaling with multiple instances}

\section{Kubernetes load balancing}

\section{Kubernetes high availability}

\section{Kubernets rolling upgrades}

\section{Kubernetes capabilities}

\section{Kubernetes deployment}

\section{Kubernetes services and workloads}

\section{Kubernetes architecture}

\section{System upgrades}

\section{Rolling updates}

\section{Canary Deployment}

\section{Recreate Deployment}

\section{Blue Green Deployment}

\section{A/B Testing}

\section{Summary}

\section{Deployment Presentation Slides}

\chapter{Technology Stack}
    
\section{Module contents overviews}
    
\section{Reference System for using tech platforms}
    
\section{Web applications}

\section{Solutions for web applications}
    
\section{Apache web server}
    
\section{Apache Web Server Architecture}
    
\section{Apache Web Server Scalabilty}
    
\section{Nginx webserver}
    
\section{NGinx architecture}
    
\section{Nginx as reverse proxy and cache}
    
\section{Web containers and spring framework}
    
\section{Jetty and Spring}
    
\section{NodeJS}    

\section{NodeJS Event Loop}
    
\section{Cloud Solutions for Web}
    
\section{Cloud Storage}

\section{Cloud CDN}
    
\section{Services}
    
\section{Services Solutions}
    
\section{Memcached}
    
\section{Memcached Architecture}
    
\section{Redis Cache and its Architecture}
    
\section{Cloud Caching Solutions}
    
\section{RabbitMQ}
    
\section{RabbitMQ Architecture}
    
\section{Kafka Architecture}
    
\section{Redis Pub/Sub}
    
\section{Cloud MQ Solutions}
    
\section{Datastores}
    
\section{Datastores Solutions}

\section{RDMBS}

\section{RDBMS Scalability Architecture}

\section{NoSQL Objectives and Trade Offs}

\section{Amazon DynamoDB}

\section{Google BigTable}

\section{BigTable Architecture}

\section{HBase}

\section{Cassandra}

\section{Cassandra Features}

\section{MongoDB}

\section{MongoDB Architecture}

\section{Analytics}

\section{Analytics Solutions}

\section{Logstash architecture}

\section{Logstash data streaming architecture}

\section{FluentID}

\section{Elasticsearch}

\section{Elasticsearch Architecture}

\section{Hadoop HDFS}

\section{Map-Reduce}

\section{Apache Spark}

\section{Stream Processing}

\section{Summary}

\section{Technology Stack Presentation Slides}
    
\end{document}