

\section{Domain, Sub-Domain and Domain Experts}
A domain is defined as a sphere of knowledge, influence or activity.
The business perspective - a domain represents the field or industry in which the business operates.
The software perspective - domain may be though of as the representation of the problem space for that software. The domain is made up of multiple sub domains.

For large and complex domains, it is virtually impossible for a domain expert of know everything about the domain (challenge accepted :D).
As a result, there are multiple subject matter experts or domain expert within a domain, mostly aligned with the sub domains in the bigger picture.

Useful questions:
- What domain do you operate in? Think business domain not technology domain.
- What are the sub domains within the domain that you operate in?
- List out the domain experts that you work with.

Summary
- Domain is defined as a sphere of knowledge, influence or activity.
- Domains are made up of multiple sub domains.
- It is not possible for a single expert to have a through knowledge of all the sub domains in a complex large domains (MVK?)
- As a result multiple domain experts are needed to support business functions.
- Mostly, these domain experts are aligned with the sub domains within the larger domain


\section{Conceptual Models, Architural Styles}
Conceptual models are defined as a representation of a system made from a composition of concepts. Key idea here is concepts.
I.e that focus is not on the physical parameters of the system.

The focus here is on the critical components that make up the system.
This information is not sufficient for the engineer in the factories to be able to build the car.

Whats the purpose?
Multiple benefits to of starting with conceptual models.
The first one is that it enhances the understanding of the designers.
As the designers put together the concept and think through it, they may find flaws in the design or the may find opportunities for improvement.
---> A good conceptual model will lead to a better product.
Second one is that it makes it easy to convey the ideas behind the concept to the stakeholders.
Third one is that conceptual models provide a point of reference to create detailed specifications.
This model contains some information about the physical aspects of the end product.
This information, combined with the models layout can act as the starting point for the designers.

The fourth one is that conceptual models provide documentation for future reference.
The first thing they do is identify the core concepts that need to be put together to design the final product.
They come up with the common terminology for the domain concepts used in the model.
They identify the different parts of the system.

Also they will identify the different parts of the overall product of the system.
Next, the designer identifies and documents the relationship between the various domain concepts, and they also identify the critical or the foundational parameters for these concepts as well as the relationships in the banking domain.

Architectural model - the visualisation of the system represented by the model.

A more formal definition of the architectural model or architecture is that is is a structured representation of a solution that meets the requirements in the problem space.
It is a high level abstraction of parts of the end solutions that presents a view or a perspective on how the required payments will be met and it assists in answering the questions posed by different stakeholders.

What is the difference between architectural model or architecture and design?

The difference is in the level of details and focus.

Architecture is mostly high level.
Provides a skeleton for the end product and the focus is more long term concepts. Components that may not change as freqeuently or never in the lifetime of the product.
The design is relatively detailed and the focus is on implementation.
This is relative. Design can also be high level but they are relatively detailed compared ot the convetional or traditional architecture models.

Summary
1 - Conceptual models - a set of concepts and their relationships
2 - Architectural models are a structured representation of solution that meets the requirements from the problem space. Does not have any implementation details
3 - Whereas design is a structured representation of a solution that has some level of implementation details in it.


\section{Modelling Techniques and Architectural Styles}
Author opinion
- Three guiding principles
- Think about the purpose of the model and the audience for whom the model is being created for. Think about the prespective and the viewpoints of the audience.
- The idea is to think about their interests and their concerns.
- Will the model that I'm creating address their concerns and answer their questions?
- Third one is level of details. How detailed should my model be so that it provides most value to the target audience.

Many techniques for creating these models (hmmmmm)

Four plus one architectural view model describes the architecture from the viewpoints of multiple stakeholders.
Central to this technique is the idea of scenarios of use cases. Think of these as requirements.
These scenarios are used as guide for creating multiple views and these scenarios are also used for validating architecture.

Logical view focuses on the functionality or the capabilities exposed by the system to the end user.
The process view, as the name suggests explains the processes in the system and how these processes interact with each other.
The development view illustrates a system from a programmers perspective and is concerned with software management. This view is also known as the implementation view.
The last one is the physical view that depicts the system form the point of view of the engineers.
It is concerned with the topology of software components on the physical layer as well as the phyiscal connetion between these components.

%TODO add four plus one diagram

Why call it four plus one?
The reasons is there are four views and plus on is for the scenarios.
A large software project involves multiple stakeholders.
They may be tens or even hundreds of stakeholders involved in the project.
These stakeholders have different set of interests and concerns.

Business Exec - Their interest is in understanding the value that the system will provide to the end user and how that end user value will translate into value for the business.
Business experts - important that the business implements the right set of processes and these processes are accurate. Their interests will be in the process.
Developers - they are concerned with the implementation of the system.
They are concerned about the deployments and the management of the software.
As a result their interest is in the development view.
Network engineer - are not concerned with he development view or the logical view or the process view.
They would like to understand how the various servers that will host the applications or the components will talk to each other over the network.
Architecture view - finally, the architect is responsible for creating these views and ensuring that these views rae providing the most value to each of these stakeholders.

This is just a small set of stakeholders.
There may be many more stakeholders playing different roles in the software initiative.

Practice
- List out the stakeholders in your organisation and think about which views will be most appropriate for them.


UML
Created by Object Management Group.
Provides a standard set of diagrams for architectural modelling and these diagrams are created by using a standard set of notations.
Vast subject.
See other resources if required.
Latest version of UML consists of 14 diagrams which can be used to created the 4 + 1 architectural view model.

Author preference.
- Use case diagram for depicting the scenarios for logical views
- Use the state diagram and the class diagram for process view.
- Use sequence diagram and activity diagram for development view
- For development view use the component diagram and package diagram.
- For physical view deployment diagram works out the best.

Software architectural style may be through of as a reusable architectural pattern which may be used as a solution to a commonly occurring problem.
There are multiple architectural styles and these architectural styles are categorised based on the key focus area.
The service orientated architecture and the message bus architecture falls in the category of architectural style that focuses on the communication between components.
The layered architecture and the object orientated architecture and design are the common styles use when the architect is focusing on the structure of the system.
The client server and the three tier architectural styles fall in the category of deployment, wherein the focus is on the deployment of the various components that make up the system.
The database centric design and the data flow diagrams are styles used where the architect is focusing on the core data within the business domain.


The last architectural design is DDD where the focus on the business domain rather than the technolgoy.

Summary
- Architects create models using different modelling techniques.
- Two such techniques.
The four plus one architectural view model and unified modelling langauge.
- The second may be combined to create very effective architectural models.
- There are multiple architectural styles.
- Each of these follow a set of common principles and they focus on specific aspects of the system.
- Architects may adopt any of these styles depending on their needs and preferences.
- Domain driven design is an architectural style in which the focus in on the business domain.


\section{Domain Models}
Reasons businesses invest in software is to solve business problems.
Business problems may be defined as current or long term challenges and issues that may prevent the business from achieving its goals.
These goals may be short term or they may be long term.
It is important for the architects to understand the business problem.
The architect must understand the domain first and for that the architects create domain models.

%----------------

A domain model is defined as organised structured knowledge of the domain that is relevant for solving a business problem
Important: Organised and structured knowledge.

The domain model consists of multiple parts.
Key concepts are the foundational concepts related ot the domain.
Domain vocabulary - consists of common terms and their definitions used by the stakeholders when they are discussing the problem space wihin that domain.
This ensures that all stakeholders have a common understanding of all the terms used in that domain.

Think of domain entities as domain objects that have a unique identity.
The attributed in the domain objects may change over the lifetime of the object.

In the real world, domain entities have a relationship with other domain entities and the model captures these relationships.

Businesses use defined processes for carrying out the operations and these processes are documented within the domain model by way of workflows and activities.

Important thing to keep in mind is that the domain model captures structured knowledge that is used for solving a business problem.
In fact the domain model may contain additional knowledge by they way of visual depictions or diagrams and textual documentation.

As the creator of the domain model, you are in control of what should go in the domain model to make it as effective as possible.

There are no special tools for creating domain model of visualisations.
May use any typical tool that supports UML modelling for textual documentations.

Stakeholders working on the domain models may decide on the tools that will work out best for them.

Key points
Domain model is organised and structured knowledge about the domain.
The purpose of the domain model is to help with creating a sutions to business problems within that domain.

The five elements.
- Domain vocabulary
- Domain entities
- Relationship between the entities
- Workflows and activities
- Key concepts


\section{Modelling Techniques and Architectural Styles}
Enterprise domain models - also known as aggregate or unified domain models.

In a complex industry it is hard to find one expert who knows everything.

To get the domain knowledge, the teams must work with these domain experts as this knowledge is mostly not documented anywhere but is in the heads of these domain experts.
Domain knowledge is organised and structured knowledge about the domain and these domain models are created by way of a process referred to as knowledge crunching.

The team receiving the knowledge form the domain expert or experts analyses the received information and knowledge and creates the domain models.

Typically the knowledge crunching process is spearheaded by the technology team that works very closely with the domain experts to create that structured domain knowledge or the domain models.

it is common for the technology team to be led by an experienced technologist. (Some kind of IT lead)

Ideally the IT lead should have some prior experience with domain modelling exercise and it is not required that they are an expert in specific technolgies.

Opinion
- Successful leads have a breadth of knowledge in multiple technologies and they are also open to learning busienss related topics.
Rest of the technology team may be composed of team members with different skill sets and roles.

Idea is that these models could be used by software development teams to build robust well documented systems.
These enterprise modesl were referred to as the unified models or aggregate models.
The idea behind all these are the saem.

Intended software development process looks something like this.
The software development teams will focus on specific areas within the enterprise domain model and then carry out the technical design.
Code and then build the final product.
In theory this sounds good but in reality this process is marred with multiple challenges.

First challenge is the creation of the model itself.
Enterprise models are inherently complex due to the scope and szie and the fact that multiple experts need to be engaged in order to create such models.
The second one is that its hard to keep the models up to data as there is not single owner for the mode and it falls on the IT team to manage the model as well as the software product..hmmmmmm

Unfortunately after a while due to product delivery priorities, models start to fall behind the actual implementations and loses its value.

The third one is that there are linguistic challenges when you try to merge together knowledge about multiple domains into one single model.

Lol - It is very common to see the same business term have different meanings in different sub domains within the organisation.
These linguistic challenges can cause big confusion for the software development team as well as the domain experts.
An important point to note is that these challenges are no applicable to enterprise level domain models but to any domain model that has to deal with complex domains.

Pros of DDD
- Domain driven design approach provides principles and patterns to address the challenges faced with developing complex domain models.

Summary
- Knowledge crunching refers to the process of creating the domain model from the knowledge gathered from the domain experts.
- There are multiple challenges with creating models for complex domains and these challenges are addressed by the domain driven design approach.


\chapter{Understanding the Business Domain}


\section{Why understand the business}
Every modern business is a technology business. For some it is the core model. For others, technology supports the core model.
Irrespective of the core business model technology is an essential part of business.
The goal of IT teams is to help the business achieve the business goals, hence IT teams must understand the business model.
Understanding of the business model by IT teams will lead to faster delivery of value to the business.
It will lead to more active participation of the IT teams in the business decisions and it will align the IT objectives with the business objectives.
The most important outcome of having teams of having teams understand the business is that business start to think of it as a trusted partner rather than just a service provider.
What this essentially means is more value to the business, move value to the customers of the business.

\paragraph{important}
Architects and developers etc who understand the business earn the trust of their peers.
I.e don't just learn new techologies, spend time understanding the business.


\section{Introduction to Business Model Canvas}
Business model canvas is a tool that helps discuss, communicate, design and understand the organisations business model.
The benefit of the business model canvas is that the entire business model may be depicted in one image.

To create the business model canvas, one has to focus on nine core areas of the business and these are organised in the form of a canvas.

Do own research

\subsection{Building blocks}
- Nine building blocks
- The first building block is customer segments. Customers are the reason the business exists, so one has to think carefully about who are the customers.
- The next building block is value proposition. What kind of value are we providing to each of these customers.
- The next one is key resources. Emphasis on key, there are many resources that our business requires but you need to think about those resources which are essential for value proposition.
- There may be multiple resources that the business needs, but think about the key ones which without the business cannot exist.
- Next one is the key partners who are suppliers of the key resources to the business.
- Next one is technology providers. Consider procuring technology from other suppliers or partners.
- Also need to consider appropriate permissions to operate. Without them a business will not be able to operate.
- Businesses need to carry out multiple activities. Underkey activities, you need to think about the activities that the business carries out to create the value for the customer.

- Customer retention is one of the most important things for any business, and to retain a customer you need to ensure that the customers are happy with the services you are providing and the relationship you have with them.
- Under the customer relationships, one has to think about the type of relationship that is offered to each of the customer segments.
- The cashflow revenue stream depicts the inflow of revenue for the business for what does.
- The cost structure has the depiction of the cash outflow. These are the expenses incurred by the business on carrying out the key activities.
- Next one is the channels by which the customers would like to be reached.

%TODO insert example canvas diagram here

