%! Author = Marjan
%! Date = 02/02/2025
% Preamble
\documentclass[a4paper, 11pt]{book}

% Packages
\usepackage{amsmath}    % For math symbols and equations
\usepackage{graphicx}   % For including graphics
\usepackage{geometry}   % For adjusting page layout
\usepackage{fancyhdr}   % For custom headers/footers
\usepackage{hyperref}   % For hyperlinks
\usepackage{listings}   % For code listing
\usepackage{lipsum}     % For placeholder text (to test layout)
\usepackage{helvet} % Helvetica as a sans-serif font alternative
\renewcommand{\rmdefault}{phv} % Set the default font family to Helvetica

% Set page margins (A4 paper)
\geometry{top=1in, bottom=1in, left=1in, right=1in}

% Set up the header
\pagestyle{fancy}
\fancyhf{}
\fancyhead[L]{System Design Notes}
\fancyhead[C]{Your Name}
\fancyhead[R]{\thepage}

% Document
\begin{document}

% Title Page
    \begin{titlepage}
        \centering
        \vspace*{2in}
        \Huge \textbf{System Design Notes}
        \vfill
        \Large Your Name
        \vfill
        \Large Date: \today
    \end{titlepage}

    \newpage

    \tableofcontents
    \newpage


    \section{Introduction}

    \subsection{OSI Model}
    Very useful when operating on different levels of abstraction in a system

    \paragraph{Level 1} - Physical

    \paragraph{Level 2} - Data link

    \paragraph{Level 3} - Network

    \paragraph{Level 4} - Transport

    \paragraph{Level 5} - Session

    \paragraph{Level 6} - Presentation

    \paragraph{Level 7} - Application

    \subsection{C4 Model}
    Very useful for handling differnt levels of abstraction

    \paragraph{Motivations: Why should I care}
    Drivers - Performance, Scalability, Fault Tolerance.
    How to measure.....latency, throughput, cost
    Consider SLA,SLO and SLI

    \subsection{Gathering System Requirements}

    \paragraph{Requirements} - Description of what we needs to be built.
    Very different when approached from a system level.
    Higher scope allows for freedom of tools and also requries a higher level of abstraction.
    Requirements are often not from an engineer or even someone technical.
    Requirements are only part of the solution.
    Client only knows the problem they need solved.
    Clarifying questions are required.

    \paragraph{Importance} - Simply build something and then fix it, wrong requirements etc, easy to fix?
    Large scale systems (i.e at this level) are big projects that cannot be changed easily.
    Many engineers invovled and many hours.
    Hardware and Software costs.
    Contracts and financial obligations.
    Reputation and brand

    \paragraph{Type of Requirements AKA Architectural Drivers}

    \paragraph{}
    Features of the System - Functional Requirements - Describe the system behaviour i.e what the system must do - Tied to the object of the system
    This doesn't determine the architecture

    \paragraph{}
    Quality Attributes - Non-functional requirements
    System properties - Scalability, Availability, Reliability, Security, Performance etc
    This does dictate the architecture

    \paragraph{}
    System Constraints - Limitations and boundaries
    Examples - Time Constraints and Deadlines, Financial Constraints, Staffing Constraints


    \section{High-Level System Design}
    \paragraph{Software Architecture Patterns} - General repeatable solutions to commonly occurring system design problems.
    Versus design patterns which are just about organising code within a single application.
    Software architectural patterns are "blueprints" for solutions that involve multiple software components.
    The purpose is to avoid repeating mistakes and anti-patterns for large scale systems.


    \paragraph{Business incentives for this approach} - One - Save time and resources for devs and the organisation
    when building a solution on a similar scale. Use already tested development practices rather than reinvent the wheel,
    something completely new can potentially take up a lot of resources.

    \paragraph
    Two - Using an existing architecture avoids the \textit{big ball of mud}. The lack of structure on the system.
    In this scenario every service talks to every other service, services are tightly coupled, information is global or duplicated
    and there is no clear scope or responsibility for any of the components.

    \paragraph
    This situation can happen due to rapid growth of the company or a lack of architecture in general.

    \paragraph
    In this situation a system can be hard to develop, maintain and scale which in turn can lead to the failure of business objectives.

    \paragraph
    Third motivation - Other developers and architects can continue working on the system and can easily carry on and stick to the same architecture.
    Everyone will know the pattern that is being followed and understand what should and should not be done when adding to the system.

    \paragraph
    These are just guidelines.
    Architecture is best defined per use case / unique situation.
    As systems evolve certain patterns that were appropriate to the system in the past may not make sense anymore.
    This is normal.
    At this point some restructuring would need to be done and a migration to a different architecture pattern should be considered.

    \paragraph
    Motivations for patterns - many companies have already been through these migrations before so best practices can be adopted to make those migrations quickly and safely.


    \subsection{N-Tier}

    \subsection{Microservices}

    \subsection{Event-Driven Architecture}

    \subsection{Big Data Architeciture}

    \paragraph{Terminology} - Data sets which are too big or too complex or are produced at too fast that exceed the capacity of a traditional application.
    Characteristics are typically volume, variety, rate/velocity.
    This data can be used to gain insights/conclusions via visualisation, querying or predictive analysis.
    Can also be used to find anomalies, analysing logs.

    \subsubsection{Batch}
    requires the storage of incoming data in a distributed database or a distributed files system. Data is never modified, it is only added to the end.
    Key principal is that the data is processed in batches or records on a fixed schedule, or a fixed number of records that we want to process.
    The schedule can be adadpted for the users/clients needs.
    Each time a batch processing job runs in can process new data and produce and up to date view of all the current data.
    This can be can be stored in a well structured and indexed dataabse that can be queried to get insights.
    Could analyse recent data or the whole data set.

    This data is not processed in real time.

    Note: This data can be used to create a Machine learning model.

    Batch processing provides the user with high availability, no downtime for users the old data view is still available.

    Batch processing is more efficient vs processing each piece of data individually. Also higher tolerance towards human error, with regards to bad code/ deployment issues.
    Batch processing can be used to perform complex data analysis of large data sets.....

    \paragraph{Drawbacks}
    Long delay between data coming in and the result we get from the processing job.
    The view is not real time which can be an issue in some use cases (Trading, 24/7 systems etc).
    Forces users to wait a long time before they can act on the insights from the system.
    May not know that the data is not in real time.

    \subsubsection{Real time processing / Streaming}
    Each new event into the system is placed into a queue or a message broker, on the other end there is a processing job that processes each bit of data as it comes through.
    After processing the processing job updates the database that provides querying capabilities for real time visualisation and analysis.
    ó

    \paragraph{Pros}
    Can respond to data immediately

    \paragraph{Cons}
    Hard to do any complex analysis in real time as a result insight may be poor compared to to batch.
    Hard to data fusion in real time, at different time points or analysing historic data. Only limited to recent data for predictions.

    \subsubsection{Lambda architecture}
    May need processes of both strategies. Lambda architecutre takes advantages of both.
    Aims to find the balaance between fault tolerance and comprhensive analysis of the data from batch processing and the low latency that we get from real time.

    In this architecture the infrastructure is divided in to three layers, btach, speed, and serving layer.

    Data that enters the system is dispatcehd into both the batch layer and the speed layer simulataneouly.
    The purpose of the batch layer is to manage out data set and be the system of records. The data in our master set is immutable and new data is appended, never modifed.
    This data is usually on a distributed files system, optimised for for storign big files containing massive amountds of data.
    The second purpose of the batch layer is to pre computer our batch views.
    Every time we run out batch processing job, it processes all the data that we have in out master data set. Once the processing is cimplete it indexes and store the data in a read only database.
    Typically, this overrides the exusting pre computedd views that we created the previous time we ran the processing job.
    Note: The batch layer aims at perfect accuracy and operates on the entire data set.

    \paragraph{Speed Layer}
    Data is sent to this layer in paralell.
    Real time stratgey is used here. All the data goes into a queue or a messdage broker and is thien pickedup as it arrives by the processing job.
    Processing job analyses the new even and adds the processed even to the real time view ready for querying.
    This layer compensiates for the latency in the batch layer.
    Unlike the batch layer it only operates on the most recent data and doesn't attempt to provied a cimplete view or make any data corrections.

    This layer exists between the last time a batch was run and the present moment/recent data.

    \paragraph{Serving layer}
    The serving layers purpose is to respond to queries and merge the data from both the batch and speed layer and update the real time views.


    \section{Components}
    \lipsum[3]

    \subsection{Load balancing}

    \subsection{Message Brokers}

    \subsubsection{Kafka}

    \subsection{API Design}

    \subsubsection{REST}

    \subsubsection{GraphQL}

    \subsubsection{RPC}

    \subsection{Data Storage}

    \subsubsection{CDN}

    \subsection{Design Patterns}
    \lipsum[4]

    \subsubsection{Scalability}

    \paragraph{Load Balancing}

    \paragraph{Pipes and filters}

    \paragraph{Scatter and Gather}

    \subsubsection{Microservices specific}

    \paragraph{Execution orchestrator}

    \paragraph{Choreography}

    \subsubsection{Extensibility}

    \paragraph{Sidecar}

    \paragraph{Adapters}

    \paragraph{Frontends for Backends}

    \subsubsection{Availablility, error handling and recovery}

    \paragraph{Rate Limiting}

    \paragraph{Retry Pattern}

    \paragraph{Circuit breaker}

    \paragraph{Dead Letter Queue}

    \subsection{Data intensive applications patterns}

    \subsubsection{Map Reduce}

    \subsubsection{Saga}

    \subsubsection{Transactional Outbox}

    \subsubsection{Materialised View Pattern}

    \subsubsection{CQRS}

    \subsubsection{Event Sourcing}

    \section{Detailed Design}
    \lipsum[5]

    \subsection{Caching}
    Redis

    \subsection{Queues}
    Kafka

    \subsection{Protocols}

    \subsubsection{TCP}

    \subsubsection{UDP}

    \subsection{Threads and Concurrency}

    \paragraph{Actor model and Akka}
    Java low level and higher level

    \paragraph{Functional}
    Scala.....

    \paragraph{Application level}
    Scaling via instances

    \subsection{Databases}

    \paragraph{NoSQL}
    MongoDB.....graph databases

    \paragraph{SQL}

    \subsection{Networks}

    \section{Performance and Scalability}

    \subsection{Testing}

    \paragraph{Gatling}

    \paragraph{Code performance}
    Hmmmm maybe JVM specific here

    \section{Distributed Systems}

    \subsection{Clusters}
    Kubernetes

    \subsection{Storage}

    \subsection{Deployments}
    Docker, Jenkins

    \section{Basic security}

    \section{Conclusion}

    \newpage

% Example of including a code snippet

    \section{Code Example}
    \begin{lstlisting}[language=Java, caption=Java Code for a Simple Cache]
public class SimpleCache {
    private Map<String, String> cache = new HashMap<>();

    public String get(String key) {
        return cache.get(key);
    }

    public void put(String key, String value) {
        cache.put(key, value);
    }
}
    \end{lstlisting}

\end{document}
