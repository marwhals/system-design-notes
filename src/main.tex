%! Author = Marjan
%! Date = 02/02/2025
% Preamble
\documentclass[a4paper, 11pt]{book}

% Packages
\usepackage{amsmath}    % For math symbols and equations
\usepackage{graphicx}   % For including graphics
\usepackage{geometry}   % For adjusting page layout
\usepackage{fancyhdr}   % For custom headers/footers
\usepackage{hyperref}   % For hyperlinks
\usepackage{listings}   % For code listing
\usepackage{lipsum}     % For placeholder text (to test layout)
\usepackage{helvet} % Helvetica as a sans-serif font alternative
\renewcommand{\rmdefault}{phv} % Set the default font family to Helvetica

% Set page margins (A4 paper)
\geometry{top=1in, bottom=1in, left=1in, right=1in}

% Set up the header
\pagestyle{fancy}
\fancyhf{}
\fancyhead[L]{System Design Notes}
\fancyhead[C]{Your Name}
\fancyhead[R]{\thepage}

% Document
\begin{document}

% Title Page
    \begin{titlepage}
        \centering
        \vspace*{2in}
        \Huge \textbf{System Design Notes}
        \vfill
        \Large Your Name
        \vfill
        \Large Date: \today
    \end{titlepage}

    \newpage

    \tableofcontents
    \newpage

    \section{Introduction}

    \subsection{OSI Model}
    Very useful when operating on different levels of abstraction in a system
    \paragraph{Level 1} - Physical
    \paragraph{Level 2} - Data link
    \paragraph{Level 3} - Network
    \paragraph{Level 4} - Transport
    \paragraph{Level 5} - Session
    \paragraph{Level 6} - Presentation
    \paragraph{Level 7} - Application

    \paragraph{Motivations: Why should I care}
    Drivers - Performance, Scalability, Fault Tolerance.
    How to measure.....latency, throughput, cost
    Consider SLA,SLO and SLI

    \subsection{Gathering System Requirements}
    \paragraph{Requirements} - Description of what we needs to be built.
    Very different when approached from a system level.
    Higher scope allows for freedom of tools and also requries a higher level of abstraction.
    Requirements are often not from an engineer or even someone technical.
    Requirements are only part of the solution.
    Client only knows the problem they need solved.
    Clarifying questions are required.

    \paragraph{Importance} - Simply build something and then fix it, wrong requirements etc, easy to fix?
    Large scale systems (i.e at this level) are big projects that cannot be changed easily.
    Many engineers invovled and many hours.
    Hardware and Software costs.
    Contracts and financial obligations.
    Reputation and brand

    \paragraph{Type of Requirements AKA Architectural Drivers}
    \paragraph{}
    Features of the System - Functional Requirements - Describe the system behaviour i.e what the system must do - Tied to the object of the system
    This doesn't determine the architecture
    \paragraph{}
    Quality Attributes - Non-functional requirements
    System properties - Scalability, Availability, Reliability, Security, Performance etc
    This does dictate the architecture
    \paragraph{}
    System Constraints - Limitations and boundaries
    Examples - Time Constraints and Deadlines, Financial Constraints, Staffing Constraints

    \section{High-Level System Design}
    \lipsum[2]

    \subsection{N-Tier}

    \subsection{Microservices}

    \subsection{Event-Driven Architecture}

    \subsection{Big Data}

    \subsubsection{Batch}

    \subsubsection{Streaming}

    \section{Components}
    \lipsum[3]

    \subsection{Load balancing}

    \subsection{Message Brokers}

    \subsubsection{Kafka}

    \subsection{API Design}

    \subsubsection{REST}

    \subsubsection{GraphQL}

    \subsubsection{RPC}

    \subsection{Data Storage}

    \subsubsection{CDN}

    \subsection{Design Patterns}
    \lipsum[4]
    \subsubsection{Scalability}
    \paragraph{Load Balancing}
    \paragraph{Pipes and filters}
    \paragraph{Scatter and Gather}
    \subsubsection{Microservices specific}
    \paragraph{Execution orchestrator}
    \paragraph{Choreography}
    
    \subsubsection{Extensibility}

    \paragraph{Sidecar}

    \paragraph{Adapters}

    \paragraph{Frontends for Backends}

    \subsubsection{Availablility, error handling and recovery}

    \paragraph{Rate Limiting}

    \paragraph{Retry Pattern}

    \paragraph{Circuit breaker}

    \paragraph{Dead Letter Queue}

    \subsection{Data intensive applications patterns}
    \subsubsection{Map Reduce}
    \subsubsection{Saga}
    \subsubsection{Transactional Outbox}
    \subsubsection{Materialised View Pattern}
    \subsubsection{CQRS}
    \subsubsection{Event Sourcing}

    \section{Detailed Design}
    \lipsum[5]

    \subsection{Caching}
    Redis
    \subsection{Queues}
    Kafka
    \subsection{Protocols}
    \subsubsection{TCP}
    \subsubsection{UDP}
    \subsection{Threads and Concurrency}
    \paragraph{Actor model and Akka}
    Java low level and higher level
    \paragraph{Functional}
    Scala.....
    \paragraph{Application level}
    Scaling via instances
    \subsection{Databases}
    \paragraph{NoSQL}
    MongoDB.....graph databases
    \paragraph{SQL}
    \subsection{Networks}

    \section{Performance and Scalability}
    \lipsum[6]
    \subsection{Testing}
    \paragraph{Gatling}
    \paragraph{Code performance}
    Hmmmm maybe JVM specific here

    \section{Distributed Systems}
    \subsection{Clusters}
    Kubernetes
    \subsection{Storage}
    \subsection{Deployments}
    Docker, Jenkins

    \lipsum[7]

    \section{Basic security}
    \lipsum[7]

    \section{Conclusion}
    \lipsum[7]

    \newpage

% Example of including a code snippet


    \section{Code Example}
    \begin{lstlisting}[language=Java, caption=Java Code for a Simple Cache]
public class SimpleCache {
    private Map<String, String> cache = new HashMap<>();

    public String get(String key) {
        return cache.get(key);
    }

    public void put(String key, String value) {
        cache.put(key, value);
    }
}
    \end{lstlisting}

\end{document}
