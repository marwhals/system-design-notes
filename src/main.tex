%! Author = Marjan
%! Date = 02/02/2025
% Preamble
\documentclass[a4paper, 11pt]{book}

% Packages
\usepackage{amsmath}    % For math symbols and equations
\usepackage{graphicx}   % For including graphics
\usepackage{geometry}   % For adjusting page layout
\usepackage{fancyhdr}   % For custom headers/footers
\usepackage{hyperref}   % For hyperlinks
\usepackage{listings}   % For code listing
\usepackage{lipsum}     % For placeholder text (to test layout)
\usepackage{helvet}
\usepackage{arydshln}
\usepackage{wasysym} % Helvetica as a sans-serif font alternative
\renewcommand{\rmdefault}{phv} % Set the default font family to Helvetica

% Set page margins (A4 paper)
\geometry{top=1in, bottom=1in, left=1in, right=1in}

% Set up the header
\pagestyle{fancy}
\fancyhf{}
\fancyhead[L]{System Design Notes}
\fancyhead[C]{Your Name}
\fancyhead[R]{\thepage}

% Document
\begin{document}

% Title Page
    \begin{titlepage}
        \centering
        \vspace*{2in}
        \Huge \textbf{System Design Notes}
        \vfill
        \Large Your Name
        \vfill
        \Large Date: \today
    \end{titlepage}

    \newpage

    \tableofcontents
    \newpage


    \chapter{Introduction}


    \section{Goals of this document}
    The goal of this document is to collate information that I think is useful in the area of software architecture.


    \section{OSI Model}
    Very useful when operating on different levels of abstraction in a system

    \paragraph{Level 1} - Physical

    \paragraph{Level 2} - Data link

    \paragraph{Level 3} - Network

    \paragraph{Level 4} - Transport

    \paragraph{Level 5} - Session

    \paragraph{Level 6} - Presentation

    \paragraph{Level 7} - Application


    \section{C4 Model}
    Very useful for handling different levels of abstraction

    \paragraph{Motivations: Why should I care}
    Drivers - Performance, Scalability, Fault Tolerance.
    How to measure.....latency, throughput, cost
    Consider SLA,SLO and SLI.
    Easier to code more complex stuff......


    \chapter{Gathering requirements}
    Asking is a user is not practice.
    A better and more methodical way of gethering requires ments of the desired system is through use cases and user flows.
    A use cases are a particular scenario/ situation in whcih the system is used to achieve a userss goal.
    A user flow is a more detailed step by step graphical representation of a use case.


    \section{Gathering functional requriements}
    In a formal way.
    First we need to identify all the actors or users in out system otherwise relevant usecases may not be caught.
    Second step is to describe all the possible use cases or scenarios in which an actor can use our system.
    Finally, the third step is to take each use case and expand it through a flow of events or interactions between the actor and the system.
    In each interaction we capture the action and take a note of the data that flows with it to and from the system.

    Multiple ways to represent this, one way is a sequence diagram which is a part of UML. Standard for viaualising system design.


    \section{Non functional requirements}
    These requirements have an effect on the software architecture.


    \section{Why redesign}
    System is functionally correct but speed, scalability, maintenance or security is too hard for reasons like number of users or data volume.
    System is functionally the same after redesign.
    Right architecture from the start is important
    Quality measures are used to measure how well the system performs and correlate with the architecture.
    Quality attributes have to be measurable and testable and this must be consistent.

    Something to note is that no single architecture can provide all the quality attributes. Certain attributes contradict each other.
    Some combinations are hard/ impossible to achieve.
    Tradeoffs are needed.

    Third important consideration of the system is feasibility of the quality attributes.
    Architect needs to ensure that the system is deliverable and what the client is asking for.
    Client could actually ask for something that is not technical or not technically possible.
    I.e expensive or not feasible.
    An example could be unrealistic latency expectations.
    Before approving a requirement, consulting with a domain expert to ensure that the requirement can actually be delivered.

%    Three important quality attributes when desigining a system
    \begin{itemize}
        \item Testability and measurablility
        \item Make tradeoffs between quality attributes over others
        \item Feasibility of quality attributes
    \end{itemize}


    \section{System constraints}
    Once functional requirements are defined, system function is known there are usually multiple ways to achieve the desired system
    Quality attributes will lead to tradeoffs and this is normal.
    System constraints are referred to as pillars of an architecture.
    System constraints provide a starting point since they are usually non-negotiable and the rest of the system has to be designed around it.

    Three types

    \begin{itemize}
        \item Technical - Being locked to hardware, cloud vendors etc, must be on prem, etc.
        \item Legal - Geography, different rules in different countries where the system resides.
        \item Business - Limited budget, strict deadlines.
    \end{itemize}

    Once constraints are identified consider non-negotiable constrains and self-imposed constraints that can be removed.
    Once a set of constraints have been agreed it is hard to move away from them.
    If further in the project timeline it is found that some of those constraints were not really constraints, then the architecture will not make sense in the future.
    Another consideration is that when certain constraints are accepted, enough space needs to be left in the architecture to move away from those constraints in the future.
    Need to avoid tight coupling with certain components or constraints to allow for change otherwise whole system could be rearchitected.


    \section{Gathering System Requirements}

    \paragraph{Requirements} - Description of what we needs to be built.
    Very different when approached from a system level.
    Higher scope allows for freedom of tools and also requires a higher level of abstraction.
    Requirements are often not from an engineer or even someone technical.
    Requirements are only part of the solution.
    Client only knows the problem they need solved.
    Clarifying questions are required.

    \paragraph{Importance} - Simply build something and then fix it, wrong requirements etc, easy to fix?
    Large scale systems (i.e at this level) are big projects that cannot be changed easily.
    Many engineers involved and many hours.
    Hardware and Software costs.
    Contracts and financial obligations.
    Reputation and brand


    \section{Type of Requirements AKA Architectural Drivers}

    \paragraph{}
    Features of the System - Functional Requirements - Describe the system behaviour i.e what the system must do - Tied to the object of the system
    This doesn't determine the architecture

    \paragraph{}
    Quality Attributes - Non-functional requirements
    System properties - Scalability, Availability, Reliability, Security, Performance etc
    This does dictate the architecture

    \paragraph{}
    System Constraints - Limitations and boundaries
    Examples - Time Constraints and Deadlines, Financial Constraints, Staffing Constraints


    \chapter{Quality attributes in large scale systems}


    \section{Performance}
    Some performance metrics
    Response time - Time between client sending a request and receive a response
    Two parts, the processing time with code and databases applying business logic.
    And the waiting time.
    The duration of time that the request or response inacitvely spends in the system.
    This time is usually spend in transit in the physical network or in software queues waiting to be handled or reach its destination.
    Waiting time is also known as latency.

    Response time can also be called end to end latency and is an importnat metric.

    Another important metric is throughput, which is the amount of data a system can process in a given interval of time.
    The more per unit of time the higher the throughput.

    \subsection{Measuring}
    Consider the end to end response time.
    Consider the distribution of response times and set a measure around it.
    Use a histogram, chart etc, statistics etc.
    Especially consider tail latency, the responses that take the most time in comparison to the rest of the values.
    This tail latency needs to be as short as possible.
    Another consideration is the performance degradation point and how fast or steep the degradation is.
    This is the point in the performance graph where the performance is starting to get significantly worse as the load increases.
    Usually this means a resource is fully utilised, could be hardware resource etc or a software queue.


    \section{Scalability}
    Load and traffic patterns never stay the same.
    Seasonal, time etc.
    Scalability is a measure of a systems ability to handle a growing amount of work in an easy and cost effective way by adding resources to the system
    Optimistic scenario is linear scalability but in practice this is hard to acheieve.
    Three ways to scale
    \begin{itemize}
        \item Horizontal scaling - Add more units of the resources that we have like multiple computers.
        Spread the load.
        No limit to scala.
        Can easily add and remove systems as required.
        Provides fault tolerance and high availability straight away.
        Not every application can support this and will require code changes.
        Changes will usually be once.
        Groups of instances require more coordination and is complex.
        \item Vertical Scaling - add more CPU, higher bandwidth etc.
        Upgrade a computer.
        No code changes required.
        This has a limit and can easily reached on a global / internet system.
        Easy to do with a cloud provider.
        This can centralise the system and fail to provide fault tolerance and availability.
        \item Organisational scaling - Add more engineers.
        Before a certain point, the more engineers, the more work gets done, the more productivity is increased.
        Eventually we get less when we add too many.
        Meetings become more frequent and crowded.
        Code and merge conflicts.
        Code base will grow too large, harder for new engineers to learn the code base.
        Testing becomes hard and slow since there is no isolation, a minor change can break everything.
        Releases become risky since they include many changes.
        This impacts team scalability and engineering velocity.
    \end{itemize}
    One solution is to separate code into separate services which their own code base and tech stack, release schedule etc.

    These scalability solutions are orthogonal to each other.


    \section{Availability}
    Bad availability - Business loses money, bad things happen on important systems.
    Availability is either the fraction of time the probability that our service is operationally functional and accessible to the user.
    That time that out system is operationally functional and accessible to the user is often referred to as the uptime of the system.
    The time-out system is unavailable is referred to as downtime.
    Availability is measured as a percentage representing the ration between the uptime and the ensure time out system is running.
    This is the total sum of uptime and downtime of the system.
    Availability is measured in percentages.
    Other metrics are Mtbf and Mttr.

    Mtbf - the mean time between failures, represents the average time out system is operational.
    This is useful when dealing with multiple pieces of hardware. Most stuff is on the cloud and this avialable upfront.

    mTOR - mean time for recovery - the average time it takes to detect and recover from a failure, which is the average down time of our system.
    Until failure is recovered from the system is essentially non-operational.

    Availability of a system is the mtbf divided by the sum of mtbf and mttr.
    Use formula to estimate availability.
%    Add equation
    Note: If we minimise the average time to detect and recover from a failure theoretically all the wy to 0 we can essentially achieve 100\%
    availability regardless of the average time between failures.
    This is not practical but detectability and fast recovery has a positive impact on availability.


    \section{Fault tolerance and high availabilty}
    Three categories of failures.
    Human error, software errors like garbage collections and hardware failures.

    Best way to achieve high availability in our system is through fault tolerance.
    This enables the system to remain operational and available to the users despite failures within one or multiple components.
    When failure happen fault tolerance will allow the system to continue operating at the same level of performance or reduced performance.
    But it will prevent the system from being entirely unavailable.

    Fault tolerance revolves around three major tactics.
    Failure prevention, failure detection and isolation and recovery.

    First thing to stop a system going down is to eliminate any single point of failure in our system.
    Best way to do this is through replication and redundancy.
    Replicas, spatial redundancy and time redundancy.

    Two strategies for redundancy and replication.

    \paragraph{}
    Active-active architecture - Requestions fo to all the replicas, replacement is available immediately.
    This allows for horizontal scalability but also coordination between all the replicas since they are taking requests.
    Not easy to keep in sync and has additional overhead.

    \paragraph{}
    Active-passive - one primary instance, passive instances take periodic snapshots.
    Lose ability to scale out system, all the requests go to one machine.
    Implementation is a lot easier, one instance is the most up to date, the rest or followers.

    Second tactic for fault tolerance - failure isolation and detection
    To achieve this we need a monitoring service to monitor the health of our instances via health check messages.
    Alternatively can listen to periodic messags called heartbeats, that should come periodically from healthy instances.

    In either strategy if monitoring service does not hear from the for a predefined duration of time, it can assume that the server is no longer avaiable.
    Can lead to false positives.
    Doesn't need to be perfect, as long as there are not false negatives.
    False negatives mean that the monitoring service failed since it didn't detect it.

    Monitoring service can be more complex
    Can monitor for certain conditions, can collect data like number of exceptions, latency etc.

    \subsection{Failure recovery}
    Third tactic for fault tolerance.
    If we can detect and recover from each failure faster than the user can notice then our system will have high availability.
    Once we detect the and isolate the faulty instance or server serveral actions can be taken.
    Stop sending any traffic or workload to that host.
    Attempt to restart it with the assumption that the problem will be fixed after a restart.
    Perform a rollback.
    i.e go to a previous version that was stable and correct. Useful for deploying updates.


    \section{SLA. SLO, SLI}
    Aggregates of quality attributes: SLA, SLOs and SLIs.

    \subsection{SLA}
    An agreement between the service provider and clients/users with regards to quality attributes like availabilty, performance and data durabilty and the time it takes to respond to failures.
    Includes penalties if those agreements are breached.

    Example penalties include: Refunds and service credits.

    Usually exists for external paying users but sometimes for free users.
    Occasionally for internal users but the penalties are minor if they exists.
    Free services don't usually publish SLAs.

    If other users rely on the service then it is important that they know the SLAs.
    Example users have their own SLAs

    \subsection{SLO}
    Service level objective - Individual goals that we set for the system.
    Each SLO represents a target value/ range of values that the service needs to meet.
    Quality attributes will make their way into the SLOs. Availability, end to end latency etc.

    SLOs are within a SLA.

    \subsection{SLI}
    Service level indicator - A quantitative measure of a SLO using a monitoring system. Logs could be used to calculate this.
    Once calculated this can be compared to the SL Objectives.

    \subsection{}
    SLA are usually created by business and legal teams.
    SLOs are defined by the architects and software engineers as well as the indicators.

    \paragraph{Considerations}
    Think about the metrics the user cares about.
    Don't measure everything.
    Define service level objectives around those metrics.
    From the SLO we can consider SL indicators to track the SLOs.

    Another consideration is the les SLOs the better.
    Too many SLOs make it hard for prioritisation.
    With a few SLOs its easier to focus software architecture around goals.

    Another - set realistic goals and allow room for error.
    Save costs and deal with unexpected issues.
    Commit to less than what can be provided.
    Important for external SLAs especially to avoid penalties.
    Internally these goals can be more aggressive.

    Another consideration, create a recovery plan when SLIs are indicating that SLOs are not being met.
    I.e - decide what to do if they system is down for long periods of time, performance degradation or bugs.
    This plan could include automatic alerts, automatic fail-overs, rollbacks, restarts, etc.
    and handbooks for certain situations.
    This avoids having to improvise in an emergency.


    \section{Performance testing}
    Placeholder for stuff to do with performance testing.....


    \chapter{API Design}


    \section{Motivations} - After gathering requirements,the system can be thought of as a black box whose behaviour has a well-designed interface.
    An interface is a contract between the engineers who implement the system and the client applications which use the system.
    Such an interface is a called an API since it is going to be called by other applications remotely through a network.
    Not to be confused with programming library.
    Applications calling the Api could be front-end clients like mobiles and web browsers, they can be other backend systems including those from other companies.
    After internal design each component will be called by other applications within our system.

    \subsection{Categories of APIs}
    There are generally three categories of APIs
    \begin{itemize}
        \item Public APIs
        \item Private or Internal APIs
        \item Partner APIs
    \end{itemize}

    \subsection{Public APIs}
    Exposed to the general public and any developer can use them and call them from their application.
    Good general practice for public APIs is requiring the users to register before being allowed to send requests and use the system.
    Allows for better control over who is using the system and how they are using it.
    This in turn provides better security.
    Can black list users who abuse the system.

    \subsection{Private APIs}
    Only exposed internally within the company.
    Allow other teams or part of our organisation to take advantage of our system and provide value without exposing the system directly outside the organisation.

    \subsection{Partner APIs}
    Similar to public APIs but are only exposed to companies and users with a business relationship.
    This can be in the form of buying a product or subscribing to a service.

    \subsection{Benefits of APIs}
    Benefits of APIs is that a client can enhance their business by using the system without knowing anything about the internal design or implementation.
    Once an API is defined and exposed clients don't have to wait until the system has been finished implementing.
    Clients can start making progres towards their integration goals.
    Once exposed, it is easier to design and architect the internal structure or out system because the API effectively defines the endpoints/the different routes in the system the user can use.

    \subsection{Desigining an API well}
    Encapsulate the internal design and implementation and abstract it away from the users/developers that want to use the system.
    Otherwise, it defeats the point of the API.
    An API needs to be completely decoupled from internal design an implementation to allow for design to change in the future, without breaking contract with the client.
    API needs to be easy to use and easy to understand, and impossible/hard to misuse on purpose.

    \paragraph{}
    This can be achieved by
    Having only one way to get certain data or perform a task rather than many.
    Having descriptive names for actions and resources, exposing only the information that the users needs and not more than that.
    Keep things consistent across our API will make using it a lot easier.
    Another good practice is keeping operations idempotent as much as possible.
    i.e an operation that doesn't have any additional effect on the result if its performed more than once.
    Idempotency is preferred since the API is going to be used over the network.
    Messages can be lost or a critical component inside out system may go down and the message was not recieved.
    Client has no idea but can resend the request with no change of effect.
    I.e same request is not processed twice.

    \paragraph{}
    Another good practice is pagination.
    Used when dealing with a large payload or dataset forming part of the reqeust.
    Without it most clients will not be able to handle such a large payload or dataset and this would result in poor user experience.
    Pagination allows for the client to request only a small segment of the response by specifying the maximum size of each response from out system.
    and an offset within the overall dataset.
    To receive the next segment, increase the offset.

    \paragraph{}
    Another good practice relates to operations that take a long time to complete.
    Asynchronous APIs.
    Client application receives a response immediately without having to wait for the final result.
    The response usually includes some kind of identifier that allows the client application to track the progress and status of an operation and eventually receive the final result.

    \paragraph{}
    Very important best practice.
    Explicitly versioning an API so that the client knows which API version they are currently using.
    Motivation behind versioning APIs is to allow for non-backward compatible changes, thus allow for two versions of an API.
    Deprecate the old once gradually with good communication to the clients who are still using it.

    \paragraph{Note}
    APIs can be defined in any way but some best practices have emerged.


    \section{Remote Procedure Calls - RPC}
    Ability to call a client application to execute a sub routine on a remote server.
    Difference is that it looks like a remote method is being called in terms of the code the developer needs to write.
    This feature is called local transparency.
    Remote or local looks the same.
    Some RPC frameworks allow for multiple programming languages so applications written in different languages can communicate.

    \subsection{How it works}
    API as well as the data types that are used in the API methods are developed using a special interface description language.
    This is framework / implementation specific.
    Effectively a schema or the communication between a remote client and a server in the system.
    After this definition using the specialised language, a code generation tool or special compiler can be used to generate two separate implementations of the API tool.
    This tool is framework specific.
    One implementation is for the client, another is for the server.
    Server side is called the server stub.
    Client side is called client stub.
    Stubs take care of all the implementation details of the remote procedure invocation.
    All custom object types that we declare using the interface description language are compiled into classes or structs depending on the language.
    Auto generate objects are referred to as Data Transfer Objects or DTOs.

    Now at run time, whenever the client application calls that particular RPC method with some parameters the client stub takes care of coding the data.
    This is called serialization or marshalling.
    After serialisation, the connection to the remote server is initialised and the data is send over to the stub.

    On the other end the stub is listening to the clients application messages and when the message is received it
    unmarshalled/serialised and then the real implementation of the method is invoked on the server application.

    Once the server is finished, the result is passed back through via marshalling and unmarshalling, to the client stub.
    The client application receives the unmarshalled response as a return value.
    This will look like it was a local method call.

    Well established pattern.
    Frameworks, implementation details and performance are the only real changes.
    API developers need to pick the right framework, define the API as well as the relevant types using a RPC framework.
    The description then needs to be published.

    The client and server in the system are decoupled.

    When the system has been designed and implemented a stub can be generated for new clients.
    New clients just need to generate their client stub to use the server method based on the published API definition.

    Using RPC, client and server is not restricted to programming language.

    \subsection{Pros}
    CConvenient - generate stubs which look just like local methods on objects.
    Communication details, and data passing details are abstracted away from the developers.
    Any failures result in just exception or error depending on the programming language just like normal methods.

    \subsection{Cons}
    RPC methods are less reliable and are slower.
    This can lead to performance issues.
    Client will not know how long Procedure calls will take.

    \subsection{Considerations}
    API design will need to consider this.
    A solution is to use asynchronous versions for slow methods which is a best practice.
    Another consideration is idempotency in the case that a message is lost/ slow

    This is generally used when backend systems are communicated to one another but frontend systems can be considered.
    RPC is a good style when providing an API to a different company vs an end user or a webpage.
    Good for different internal components within a large scale system.

    The RPC API style is also a good way to abstract away the network communication and focus on the actions the client wants to perform on the server.

    This would be a bad fit if we want to use browser cookies or headers. There are other styles for this.
    RPC revolved more around actions rather than data and resources (Think CRUD applications).


    \begin{itemize}
        \item gRPC
        \item Apache Thrift
        \item Java Remote Method Invocation (RMI)
    \end{itemize}


    \section{Representational State Transfer - REST}
    Set of architectural constraints and best practice for web APIs. It is just a style allowing / making it easier to achieve / improve quality attributes.
    A RESTful API meets a certain traits.
    More resource orientated.
    Encapsulates the different entities in a system and allows for the manipulatoin of those resources through only a small number of methods.
    In a REST api a client requests a named resource and the server responds with the current state of that resource.
    Usually done using HTTP. The resource can be implemented in a completely different way, only a representation of the resource is sent.
    Resource is just an abstraction.
    Interface is more dynamic, actions are not statically defined like in RPC.
    This is through a concept called HATEOAS - hypermedia as the engineer of the application state.
    This is achieved by accompanying a state representation reasoner to the client with hyper media links.
    The client can follow these links and progress its internal state.

    \subsection{Achieving quality attributes}
    Important requirement of a system that provides RESTful API is that the server is stateless and does not maintain any session information about the client.
    Each message should be served by the server in isolation without any information about previous requests.
    This allows for high availability and scalability.
    If there is no session information high groups of servers can be used and the load can be spread. Client will no notice.
    Cacheability - Important requirement.
    This means the server either implicitly or explicitly defines each response as cacheable or non-cacheable.
    Allows the client to eliminate the potential round trip to the server and back if they response is cached somewhere closer to the client.
    This will also reduce system load.

    \subsection{Resources in RESTful APIs}
    Each resource is named and addressed using a URI.
    Resources are organised in a hierarchy where each resource is either a simple resource or a collection resource.
    The hierarchy is represented using forward slashes.
    A simple resource has a state and optionally can contain sub resources.
    A collection resource is a special resource that contains a list of resources of the same type.

    \subsubsection{Naming resources - best practices}
    Use nouns only.
    Provides distinction between actions and the resources the actions will be taken on.
    Make a distinction between collection resources, and simple resources using plural and singular.
    Use clear and meaningful names.
    This will make the API easier to use for developers, and avoid mistakes and incorrect usages.
    Final practice.
    Use Unique, URL friendly names for resource identifiers for usage via the web.

    \subsection{Methods and operations}
    REST API limits the number of methods we can perform on each resource to a few predefined options.
    Creating, Updating, Deleting and getting the current state.
    When the resource is a collection resource, getting its state usually means getting a list of its sub-resources.

    Since RESTapis are commonly implemented using HTTP the operations are mapped to HTTP methods as follows.

    \begin{itemize}
        \item POST - Create an existing resource.
        \item PUT - Update an existing resource.
        \item DELETE - Delete an existing resource.
        \item GET - Get the state of an existing resource.
    \end{itemize}

    Additional can be defined but the situation is un common.

    \subsection{HTTP Semantics}
    Guarantees from HTTP semantics.
    GET is considered safe i.e it will not change the state of the resource.
    GET, PUT, DELETE are idempotent. i.e same result if you apply those operations multiple times.
    GETs are considered cacheable by default while responses to POST requests can be made cacheable by setting the appropriate HTTP headers sent as part of the response to the client.
    This feature allows us to conform to the cachabilty requirements of a REST API.
    When the client needs to send additional information to the system via a POST or PUT command JSON can be used.
    XML is also usable.

    \subsection{Step to creating a RESTful API}
    \begin{itemize}
        \item - Identify entities in a system to serve as resources
        \item - Mapping the entities to URIs and organising the hierarchy of resources based on their relationships
        \item - Choose a representation for each resource. Can use JSON for this and include links for HATEOS
        \item - Final step - assign HTTP methods to actions.
        \item
    \end{itemize}


    \section{GraphQL}

    
\section{GraphQL}

Dedicate this part to graphQL


    \chapter{Large Scale Systems Architectural Building Blocks}


    \section{DNS, Load Balancing, GSLB}
    Basic role, balance the traffic load amongst a group of servers in a system.
    Helps with achieve horizontal and high scalability of a system when running an application on multiple computers.
    Without a load balancer the client application will need to know the addresses of the computers hosting the application.
    This tightly couples the client application to out system's internal implementation and this makes changes hard.
    Load balancers provide an abstraction as well as avoid overloading a single server with traffic.
    This abstraction makes the application look like a single server.

    \subsection{Quality attributes from a load balancer}
    Load balancers provide
    High scalability -> horizontal scaling is easy
    Even easier in the cloud.

    Provides high availability.
    Load balancers can be configured to stop sending traffic to servers which are not healthy.

    Performance - load balancers add a little bit of latency and increase the response time but is worth the tradeoff for increased performance in terms of throughput.
    The load balancer can cater for as many backend servers as desired with some reasonable limitations.
    Throughput is much large than what would be possible with a single server.

    Load balancers helps us achieve maintainability, since we can add, remove, upgrade servers in rotation without disrupting the client/ user.
    This would be done on a subset of the servers leaving another subset running.

    \subsection{Types of load balancers}
    DNS - internet infrastructure that maps human readable URLs to IP addresses that can be used by network routers to route to individual computers on the web.
    A single DNS record doesn't have to be mapped to a single IP address.
    They can be configured to return a list of IPs corresponding to different servers.
    The list returned may be in a random order.
    When receiving a list most client applications just pick the first one in the list.
    Technically this can be seen as load balancing
    Cheap and super simple but DNS doesn't know the health of servers.
    Requests may go to a down server and the DNS does't know.
    List of IP addresses changes only so often and si based on the time to live that was configured for that particular record.
    Additionally this list of addresses that a particular domain is mapped to, can be cached in different locations such as the clients computer.
    That makes the time between a server going down and the point that the requests are no longer sent to that server even longer.

    Another drawback of the DNS load balancing strategy is that the round robin method doesn't consider the hardware resources of different servers.
    Some servers may have more powerful hardware than others, nor can it detect that one server is overloaded than others.
    Another drawback.
    The appliicatoins gets the IP address of all the servers, exposing implementation details whichi make the system less secure.
    Nothing stop a malicious client application from sending requests to a single IP address in order to overload \textbf{it}

    \subsection{Addressing these drawbacks}
    Two options.
    Hardware load balancers and software load balancers.
    All communication between the client and servers are done via both these types of load balancers.
    Individual servers, and IP addresses are hidden behind the load balancer and not exposed to the users making the system more secure.

    These load balancers can perform health checks on the servers and can detect if one has became unresponsive.

    BOth can balance load more intelligently across servers taking into account the different hardware application instances are running on.

    They can be used to balance requests from users, but they can also be used inside the system to create an abstraction layer between services.

    Superior to DNS in terms of load balancing but they are usually colocated with the group of servers they balancer the load on.
    If a load balancer is too far away from the actual servers, we are adding a lot of extra latency since all communication both to and from the application has to go through the load balancer.

    When running a system in multiple geographical locations having a load balancer between them will sacrifice the performance for at least one of these locations.

    These load balancers do not support the DNS resolution, so a DNS solution will also be required.

    \subsection{Global Server Load Balancer - GSLB}
    A hybrid between a DNS service and the hardware or software load balancer.
    Typically can provide a DNS service.
    In addition it can make more intelligent routing decisions.
    On on hand, the GSLB can figure out the users location based on the origin IP inside the incoming request.
    a GSLB service has similar monitoring capabilities to typical software or hardware load balancer.
    It knows the location and the state of each server we register with out GSLB.

    In a typical large-scale system deployment, those servers are load balancers located in different data centres in different geographical locations.
    GSLB may just return the locations of the nearest load balancer.
    The user will use that IP address from that point on to communicate with the system in that data centre through a collocated hardware or software loadbalancer.

    \subsubsection{Extra notes}
    They can be configured to route traffic on a multiple strategies.
    Not just geogrpahic location.
    Since they know the health of different data centres, they can route traffic based on current traffic or load on each data centre.
    Or based on estimated response time, or bandwidth between the user and that particular data centre.

    This can allow for the best performance possible for each use regardless of their geographical location.

    They are also important in disaster recovery situations.
    If there is an issue in one data center users can be routed to another.
    This provides high availability.

    To prevent a load balancer from being a single point of failure, we can register all the addresses with the GSLB DNS service or any other DNS service.
    This allows clients to get a list of all the load balancers and either pick one or send their request randomly.

    \subsection{Load Balancing Solutions}
    \begin{itemize}
        \item - HAProxy
        \item - NGINX
        \item - AWS - Application (Layer 7), Network (Layer 4), Gateway Load Balanacer, Classic Load balancer (Layer 7 and 4)
        \item - GCP %TODO add later
        \item - Azure %TODO add later
    \end{itemize}

    \subsection{GSLB Solutions}
    \begin{itemize}
        \item - Amazon Route 53
        \item - AWS Global Accelerator
        \item - GCP stuff
        \item - Azure stuff
    \end{itemize}


    \section{Message Brokers}
    A building block for asynchronous architectures.

    \subsection{Drawbacks of synchronous communitcation}
    Both applications that establish communication have to be healthy and have to remain healthy while the transaction is completed.
    Easy when the messages are short.
    Becomes more complex with larger messages.
    There is no leeway in the system to absorb spike in traffic or load.
    Horizontal scaling is not an option since a transaction/ processing may take a long time on the system.
    The solution to this is to use a message broker.

    \subsection{What is a message broker}
    A queue data structure to store messages between senders and receivers.
    Used internally in a system, not to be exposed/interacted with by a user/client.
    Can provide additional functionality in addition to storing a buffering messages.
    Can perform message routing, transformation validation and even load balancing.
    Unlike load balancers, message queues decouple senders from receivers by providing their own communication protocols and APIs.
    They are a fundamental building block in any kind of asynchronous software architecture.

    When we have two services communicating with each other via message broker, the sender doesn't have to wait for confirmation from the reciever after it sends the message.
    The receiver doesn't even have to be available to receive the message when the message is sent.
    Very useful in breaking a service into multiple services.
    Another important benefit that message services provide is buffering of messages to absorb traffic spikes.

    Many message brokers additionally off the published subscribe pattern, where multiple services can publish messages to a particular channel and multiple services can subscribe to that channel and get notrified when a new event is published.
    This makes it very easy to add services that bring in additional functionality, with out modifying the system.

    \subsection{Quality attributes from adding a message queue}
    Adds fault tolerance, since it allows different services to communicate with one another while some may be unavailable temporarily.
    Message brokers prevent messages from being lost, a characteristic of a fault tolerant system.
    This in turn helps provide higher availability for our users.
    Since a message broker can queue up messages when there is a traffic spike, it allows our system to scala to high traffic withouth modifying the system.

    \paragraph{Drawbacks}
    Performance - Introduces latency through the indirect communication between services.
    Not much of an issue in most systems.

    \subsection{Message broker solutions}
    \begin{itemize}
        \item Kafka
        \item RabitMQ
        \item AWS SQS
        \item GCP
        \item Azure %TODO add more info...?
    \end{itemize}


    \section{API Gateway}
    The problem being solved.....split the monolith in separate services....now there is a lot of service duplication and performance overhead.
    Each separate service needs to implement its own security, authorisation, authentication etc....
    To solve this decouple the client application from the internal organisation of the system and simplify out external API.
    This can be done via another abstraction called the API Gateway.
    An API Gateway is an API management service that sits between the client and the collection of backend services.
    API Gateway follows a software architectural pattern called API composition.

    In this pattern, we compose all all the different APIs of our services that we want to expose externally into a single API.
    This single API can be called by applications, by sending requests to one service.
    This vs sending multiple requests to different services to achieve a task.

    Provides an abstraction between the client and the rest of our system.

    \subsection{Pros}
    Allows for internal changes easily for API consumers.
    Can consolidate security issues into one place.....the API gateway.
    Bad requests get stopped at the gateway.
    Can allow a user to perform different operations depending on his permissions and role.
    Can implement rate limiting at the API Gateway to stop DDOS

    Can also improve performance of the system by saving a lot of overhead.
    Only have to do certain actions once.
    Can stop the user from making multiple requests to different places.
    This is request routing.
    With an API gateway the client makes a single request and all the backend services will have their responses aggreagatesd into a single response.

    Another performance gain is from caching certain responses for particular requests.
    Avoid making request to the backend systems.

    Another gain is monitoring and alerting is easier.
    By adding monitoring logic into the API gateway real time information can be gained on the traffic and load of they system..
    Can create alerts based on traffic variations.
    Improves observability and availability.

    Also allows for protocol translation from one place.
    Send JSON but work with RPC internally etc.
    Could communicate with legacy services the support older protocols.
    Some systems maybe reluctant to change this.
    Instead different API formats can be catered for at the API Gateway.
    I.e read some external format and translate it for the internal system.

    \subsection{Best practices and anti patterns}
    Do not include business logic here.
    Main purpose is API composition and routing of requests to different services.
    Those services are the services that make the business decisions and perform the actual tasks.
    If you add business logic to it that service will end up doing all the work.
    I.e a single service.
    In turn this will become an unmanageable amount of code.
    One of the problems we want to solve by splitting a service into multiple services.

    Next consideration, is that since all traffic goes through it, the API gateway may become a single point of failure.
    This can be solved by adding multiple instances of the API gateway service and placing them behind a load balancer.
    This solves scalability, availability and performance aspect.

    Another thing to consider is deployment.
    A bad release/ bug can crash the APi gateway service and the entire system can become unavailable to clients.
    Human error needs to avoided and deployments require thought.

    Additional latency is added as well since there is another service that request must go through before performing business actions.
    Avoiding the API gateways is an anti-pattern that should be avoided even though it may optimise the request processing.

    \subsection{API Gateway Solutions}
    \begin{itemize}
        \item Netflix Zuul (Open Source)
        \item Amazon API gateway
        \item GCP - Apigee , Google Cloud plateform API Gateway
        \item Azure
    \end{itemize}


    \section{Cloud Delivery Network Solutions and cloud}
    Can be considered to be more of a service.
    The problem being solved.
    Even with GSLB there still latency between the end user and the locations of the hosting server.
    Each request has to go through multiple hops over the network between different routers etc adding even more latency.
    Users will abandon a website if it takes to long to load. (Think 3 seconds plus)
    Can improve system performance etc but its content that needs to be closer to the users rather than business logic.

    \subsection{What it is}
    A globally distributed network of servers located in strategic places with the main purpose of speeding up the delivery of content to the end users.
    Solves bad user experience.
    CDNs cache website content on their servers.
    Referred to as edge servers. They are physically close to the user and more strategically located in terms of network infrastructure.
    Allos for the transfer of content much quicker to the user and improve the perceived system performance.
    Can be used to deliver webpage contents and assets including video streams.
    Both live and on demand.
    Very widely used.

    Results in faster page loads, improves system security and helps protect against DDOS since malicious requests won't go to our system.
    They will be distributed amongst a large number of servers hosted by the CDN provider.

    In addition to physical closeness, they can also be hardware optimised. I.e better hardrives, CPU etc.
    Can also reduce bandwidth by compressing content delivered over a network.
    Examples, GZIP and JavaScript minification.

    \subsection{Strategies when integrating with CDNs}
    Pull strategy - tell CDN provider which content we want on our website to be cached and how often this cache needs to be invalidated.
    This can be configured in the time to live property on each asset or type of asset.
    In this model first time a user requests a certain asset the CDN will have to populate its cache by sending a reqeust to a server in our system.
    Once that asset is cached on the CDN, subsequent reqeust by users will be served by the edge servers directly.
    This saves the network latency associated with the communication with out servers.
    When requesting an asset that has already expired, CDN will check for a new version.
    If it has not changed the CDN will refresh the expiry time for that asset and serve it back to the user.
    Otherwise, if a new version is available the CDN will receive the new version instead of the old one to the user.

    Push strategy - Manually or automatically upload or publish the content that we want deliver through a CDN.
    WHen the content changes, we are responsible for republishing the new versions to the edge servers.
    Some CDN providers support this model directly, other enable the strategy by setting a very long TTL for our assets so the cache never expires.
    When we want to publish a new version, we simply purge the content from the cache which forces the CDN to fetch that content from the servers whenever a user reqeusts that content.

    \subsection{Choosing the right strategy}

    \subsubsection{Advantages of pull model}
    Lower maintenance on our part.
    Once configured which assets need to cached by the CDN and how often they need to expire nothing needs to be done to keep them up to date.
    Everything at that point will be taken care of by the CDN provider.

    \paragraph{Drawbacks}
    FFirst time there will be a longer latency as the CDN cache is populated from the server.
    If the time to live is the same for all assets there may be frequent traffic spikes when those assets expire at the same time.
    This would result in a large number of requests from the CDN to refresh its cache at one time.

    \paragraph{}
    Add to the availability of the system but the servers still need to maintain availability otherwise the CDN won't be able to pull the latest version of assets.

    \subsubsection{The push strategy}
    Good if the content doesn't change to frequently.
    Push to the CDN and then traffic will go the CDN servers.
    Will reduce traffic to out system and reduces the burden on out system to maintain high availability.
    Even if they system goes down, users can still get data from the CDN and won't be affected by our systems internal issues at all.
    If content does change frequently then we have to publish new versions to the CDN otherwise users will get stale and out of data content.


    \chapter{Data storage at Scale}
    Motivations: Choosing the right databases amongst many options.


    \section{Relational Databases and ACID Transactions}

    \subsection{Relational Databases}
    Data is stored in tables.
    Each row in a table corresponds to a single record and all the records are related to each other through predefined columns they all have.
    Each column in a table has a name, a type and optionally a set of constraints .... NULL etc.
    The relationship between all records inside a table is what gives this type of database the name relational database.
    Structure of each table is decided ahead of time and is referred to as the schema of the table.
    We know what each record in the schema must have because because it is predefined.
    We can use a very robust language to query the data (analyse and update) in the table.
    This is SQL.
    Different implementation add their own features to this language.
    Oracle etc....
    Majority of operations are the same across all relational databases.
    Proven way of storing structured data.
    Avoids data duplication when memory is usage has to be considered (large scale)
    Use joins to combine information from multiple tables without duplication.

    \subsection{Advantages}
    Carryout complex and flexible queries using SQL. Use for analysis
    Save memory because multiple tables can be joined ..... save costs.
    Easy to reason about, very natural for humans.
    No sophisticated knowledge requried.
    Most importantly, provides ACID transactions.

    \begin{itemize}
        \item Atomicity
        \item Consistency
        \item Isolation
        \item Durability
    \end{itemize}

    In the context of transactions the sequence of operations should look like a single operation externally.
    Relational databases guarantee atomicity of transactions.
    Atomicity - transactions appear once or not at all.
    Consistency - guarantees that transaction that has been committed will be seen by all future queries and transactions.
    Also guarantees that data constraints that are set for the data, are met.
    Isolations - relates to concurrency, if two transactions are taking place the second transaction will not see an intermediate state. They are separate.
    Durability - once a transaction is complete its final state will persist.

    \subsection{Disadvantages}
    Rigid structure enforced by database schema enforced by the schemeas
    Schemas has to be designed ahead of time before the table can be used.
    Future changes to the schemas can lead to down time.
    Changes ideally should be avoided and not done at all.
    Through planing is required for this.
    Harder and more costly to scale because of their complexity.
    Providing SQL and ACID transactions is not straight forward and thus relational databases are harder to maintain.
    Due to ACID guarantees, reads are slower to perform compare to other types of databases.

    Different implementations have different performance optimisations and guarantees.

    Generally, relational databases are slower than non relational ones.

    When choosing one consider pros/cons and the use case.
    I.e is the data related, are read important.


    \section{Non-Relational Databases}
    NoSQL databases generally allow for the logical grouping or records without them having to conform to some kind of schema.
    Can easily add some kind of attributes to the different records without redesign or effecting other records.
    \textit{Most} languages do not have tables as a structure....external libraries required.
    NoSQL more typical computer science data structure (arrays, trees etc).
    This eliminates the need for object relational mappings/ ORM (Hibernate) to translate business logic for storage in a database.....

    Relational Databases are designed for efficient storage (low memory availability/ expensive memory).
    NoSQL databases are typically optimised towards faster queries.
    Different types of non-relational databases are optimised for different types of queries based on the use case.

    \paragraph{}
    Issues with flexible schemas
    Loss of ability to easily analyse those records since each record can have different structure and data.
    Joining becomes very hard, these operations are often not supported by NoSQL databases or are hard.
    Each NoSQL database supports a different set of operations and different set of datastructures.
    ACID transactions guarantees are rarely supported by non-relational databases (see exceptions).

    \paragraph{}
    Three types of NoSQL databases. Categories are somewhat blurry.

    First type is simple key, value store. We have a key that uniquely identifies a record and value that represents the data.
    The value can be anything, primitive or a binary blob.
    Can be though of as a large scale hash table/dictionary with few constraints on the type of value that can be held for each key.
    Good for caching pages or for quick fetching and ....easy.... querying.

    Second type - A document store. Collections are stored as documents. Documents have a bit more structure.
    Each document can be thought of as an object with different attributes.
    Those attributes can be different types.
    Similar to classes and fields.
    Documents are easily mapped to objects inside a programming language. Think JSON, YAML, XML

    Third Type - A graph database, an extension of a document store but with additional capabilities to traverse, link aand analyse records more efficiently.
    These types of databases are particularly optimised for navigating and analysing relationships between records in a database.

    \paragraph{Use Cases}
    Fraud detection.
    Recommendation engines.

    \subsection{When should they be used}
    Look at use case and analyse what is required from the database and what can be compromised.
    Non relation databases are better when it comes to query speed.
    Good for caching.
    Can store common query results that correspond to user views/pages.
    In memory key/value stores are optimised for this.
    Real time big data is a good use case, since relational databases are too slow and not scalable enough.
    Another use case \ldots data is not structured and different records can contain different attributes.

    Key-Value stores
    \begin{itemize}
        \item Redis
        \item Aerospike
        \item Amazon DynamoDB
    \end{itemize}
    Document Stores
    \begin{itemize}
        \item Cassandra
        \item MongoDB
    \end{itemize}
    Graph Databases
    \begin{itemize}
        \item Amazon Neptune
        \item NEO4j
    \end{itemize}


    \section{Techniques to improve performance, availability and scalbilty of databases}
    Three techniques to improve the scalability, performance and availability of a databse.

    \subsubsection{}
    Technique one - Indexing - Speeds up retrieval operations and locate the results in a sublinear time.
    Without indexing those operations could require a full table scan, which is bad for large tables.
    Example operations
    Sorting, search.
    If performed often can lead to a bottleneck

    An index is helper that we create from particular column or group of columns.
    When the index is created from a single column, the index table contains a mapping from the column value to the record that contains the value.
    Once that index table is created, we can put that table inside a data structure (good strucutres - B-Trees(or any self balancing tree), or a Hashmap).
    This will keep the values sorted and thus makes searching more efficient (think big O complexity).
    Searching is log n.
    Sorting is n log n.

    Composite indexes can be created.

    %TODO add reminder on normal form........for relational databases

    Optimising for one operation can lead to a tradeoff elsewhere.
    Indexing increases memory, decreases the speed of writes but the increases the speed of reads.
    Writing records become slower because each time a write/ new record is performed the index table must also be updated.
    Indexing is also used in non-relational databases to speed up queries.

    \subsubsection{}
    Database replication - single database, single point of failure.
    Solution, replicate data and run multiple instances of the database on different computers.
    This increases fault tolerance, availability.
    One replica goes down another can take its place and business is not effected.
    Queries can continue going to available replicas while the faulty replica is fixed.

    Get better performance and better throughput.
    Better throughput by distributing our queries across multiple computers/ replicas.

    \paragraph{Tradeoffs}
    Introduces higher complexity especially with regards to write, update and delete operations.
    (Non idempotent???)
    Making sure concurrent modifications to the same records don't conflict with each and providing predictable guarantees in terms of consistency and correctness is not a trivial task.
    Distributed databases are very hard to design, configure and manage....especially at scale.
    Requires knowledge of distributed systems....

    Database replication is widely supported by most modern databases
    Non-relation databases support replication (high availability, large scale) out of the box as they were designed with that in mind.

    Support for replication with relational databases varies amongst different implementations.

    \subsubsection{Database partioning}
    Also known as sharding.
    Split data amongst different database instances for better performance.
    Each instance will run on a different computer typically.
    More computers, more data.
    Queries which use different partitions can be done in paralell.
    We get both better performance and better availability.
    Sharing effectively turns the database into a distributed database.
    Add complexity and overhead since routing is required for the right shard/partition as well as making sure one shard does not become too large.

    Database sharding is a common feature in most non-relational databases.
    Since by design they they decouple different records from each other.
    Storing records on different computers is a lot more natural and easier to implement.

    For relational databases, it depends on the implementation.
    Relation database queries usually involve more than one record/ are more common.
    Splitting these across multiple machines is more challenging to implement in a performant way while supporting thigns like ACID transactions or table joins.
    When choosing a relational database for high volume of data use case check for partitioning support.
    Partitioning can also be used to split infrastrcture like compute instances and redirect traffic to different machines.
    Can also do this based on mobile/browser etc running the same application.

    This routing allows for understanding which users are affected during an outage.

    \paragraph{Note}
    These three techniques are orthogonal to each other. Often used together in real user use cases.


    \section{Brewers CAP Theorem}
    %TODO find a good diagram for this

    ----- In the presence of a network partition, a distributed database cannot guarantee both consistency and availability.

    The scenario of database instances not being able to communicate with one another.
    When an isolated server cannot communicate with its replicas it can either return an error or produce inconsistent data.

    Note: No network partition then no tradeoff and both consistency and availablitily can be offered.

    \begin{itemize}
        \item Consistency - Every read request received the most recent write or an error regardless of which instance is serving when considering a network partition.
        \item Availability - Every request receives a non error response without the guarantee that it contains the most recent write.
        Occasionally different clients may get a different version of a record, some may be stale but all requests return a successful value.
        \item Partition tolerance - means that the system continues to operate despite an arbitrary number of messages being delayed or lost by they network.
    \end{itemize}


    The theorem basically states one of these must be dropped when we are working with a database that may have to deal with a database partition.
    Could have a single database to guarantee consistence and availability....ie no network involved with the database....but this would not scale.
    When architecting for partition tolerance we either have to drop availability or consistency.
    Depends on the use case.

    The choice is not so distinct, more of a spectrum.
    More availability less consistency, less consistency more avialabiltiy. Depends on the tolerance requirements of the application.

    Important to get these tradeoffs in the architectural design phase, when we formalise non-functional requirements.


    \section{Scalable unstructured data storage}
%TODO where does pine cone and vector data bases fit in...........
    Unstructured data, data that doesn't follow a model or schema. Non relational data still has keys and values and structure.
    Consider binary files.
    Databases may allow for storage but they are not optimised for it.
    Limit on the object size with DBs otherwise we would encounter scalability and performance problems.
    Use cases
    - Disaster recovery
    - Archiving
    - Web hosting media like videos etc
    - Collecting data points for machine learning purposes

    Huge datasets, storage solution needs to be scalable.
    Each object can also be big.

    First solution - Distributed file system - network of storge devices
    We can get different replication consistency, and auto healing guarantees based on the file system.
    Main feature is that files are stored in a tree like structure.
    Benefit of storing unstructured data like this is that special APIs are not needed.
    Can easily modify these files.
    Performance intensive operations such as big data analysis or transformations operations on data are very fast if we do it on a distributed file system.
    ----Very useful for machine learning projects.

    Limitations of Distributed file systems
    - limited in the number of files we can create which is scalabilty issue with relatively small files like images.
    - easy access to those files via web api is hard, Additional abstractions would be required.


    Another solution - use an object store
    - A scalable storage solution designed for unstructured data at an internet scale.
    - Can scale linearly like a distributed file system by adding more storage devices.
    - Unlike a distributed file system we have virtually no limititation on the number of binary objects we casn store in it
    - High sizes on the size of a single object....can be terrabytes.
    - Good for archiving and backups
    - Features of typical object stores can include HTTP rest API that makes them effective for storing multi media content - images etc that can be linked to a webpage
    - Object versioning is another feature, on a file system we would need to use an external versioning system.

    Differences between object store and a distributed file system.
    - No directory hierarchy
    - Flat structure called buckets
    - main abstraction is an object
    - Typically key value pairs. value being the content. Additional key value pairs exist for meta data like file type and size.
    - Objects can include an access control list who can access who has read/write permissions.
    -

    Commonly provided by cloud providers.
    Typicall broken into classes with each class prvoding different prices and SLA guarantess.
    Top tier is usually availability of a high percentage, offers lowest latency and highest throughput
    Top tier is best for data that will be frequently accessed lke content videos, images etc.
    Middle tiers have lower guarantees for viability and are cheaper.
    - These options have limited performance and sometimes have limits frequency of usage as well.
    - Good for data backups, i.e data not needed often
    Lowest tier is good for longer term archiving.
    Usually cheap and used for specific use cases.

    \subsection{Cloud vs OpenSource vs third party managed}
    Cloud may not be an option for budget, legal or performance constraints.
    There are options for on premise storage devices.
    Some follow the same APIs as some cloud vendors, so they can easily be used in hybrid cloud environments (cloud + private data centre).
    Can use the same object ID to store data in both locations.


    Just like distributed file systems, object stores use data replication although this is abstracted away.
    Ensure physical storage loss will not result in loss of actual data.

    \paragraph{Object store downsides}
    Data is immutable, no edits can only be replaced with a newer version.
    Performance implications --- storing large documents that require ammends would not be feasble
    Needs a special API or a rest API unlike a distributed file system.


    Distributed systems are the preferred storage solution over object stores for high throughput solutions.

    \subsection{Scalable Unstructured Data Storage Solutions}
    Cloud based
    \begin{itemize}%TODO look into differences
        \item Amazon S3
        \item GCP
        \item Azure
        \item Alibaba
    \end{itemize}

    Open source and third party
    \begin{itemize}
        \item OpenIO - software defined
        \item MinIO - S3 and native to kubernetes
        \item Ceph - \textbf{Open-source} reliable and scalable.
    \end{itemize}


    \chapter{High-Level System Design}

    \paragraph{Software Architecture Patterns} - General repeatable solutions to commonly occurring system design problems.
    Versus design patterns which are just about organising code within a single application.
    Software architectural patterns are blueprints for solutions that involve multiple software components.
    The purpose is to avoid repeating mistakes and anti patterns for large scale systems.

    \paragraph{Business incentives for this approach} - One - Save time and resources for devs and the organisation
    when building a solution on a similar scale.
    Use already tested development practices rather than reinvent the wheel,
    something completely new can potentially take up a lot of resources.

    \paragraph
    Two - Using an existing architecture avoids the \textit{big ball of mud}.
    The lack of structure on the system.
    In this scenario every service talks to every other service, services are tightly coupled, information is global or duplicated
    and there is no clear scope or responsibility for any of the components.

    \paragraph
    This situation can happen due to rapid growth of the company or a lack of architecture in general.

    \paragraph
    In this situation a system can be hard to develop, maintain and scale which in turn can lead to the failure of business objectives.

    \paragraph
    Third motivation - Other developers and architects can continue working on the system and can easily carry on and stick to the same architecture.
    Everyone will know the pattern that is being followed and understand what should and should not be done when adding to the system.

    \paragraph
    These are just guidelines.
    Architecture is best defined per use case / unique situation.
    As systems evolve certain patterns that were appropriate to the system in the past may not make sense anymore.
    This is normal.
    At this point some restructuring would need to be done and a migration to a different architecture pattern should be considered.

    \paragraph
    Motivations for patterns - many companies have already been through these migrations before so best practices can be adopted to make those migrations quickly and safely.

    \subsection{N/Multi-Tier Architecture}

    \paragraph{Intro}
    Organise the system into multiple physical and logical tiers.
    Logical separation limits the scope of responsibility.
    Physical tier allows each tier to be deployed, upgraded, scaled separately by different teams.

    \paragraph{}
    Note: Multi-tier and Multilayer are two different concepts.
    Multi layered architecture usually refers to the internal sepration inside a single application into multiple logical layers or modules.
    Even if the application is logically seprated into multiple layers at runtime it will run as a single unit and will be considered a single tier.

    Multi-tier architecture - applications on each tier run physically run on different infrastructure.

    \paragraph{}
    Benefits of logical and physical separation-Allows us to develop update and scale each tier independently.
    Restrictions in this architectural pattern.

    \paragraph{}
    First restriction is that each pair of applications that belong to adjacent tiers communicate with each other using the Client-server model.
    REST etc.
    Second restriction - discourages communication that skips through tiers.
    This keeps the tiers loosely coupled with each other allowing for easy changes between tiers withouth effecting the entire system.

    \paragraph{Three tier - Common variation}
    \hrule
    Top level contains UI aka presentation tier.
    Display information and take user input through a GUI\@.
    No business logic normally, this tier usually runs in the client browser.
    I.e. JavaScript etc.
    The code here should be assumed to be visible and accessible to the user.
    Bad place for business logic doing so is considered an anti-pattern.
    \hrule
    Application Tier / Business Tier/ Logic Tier.
    Provides the logic gathered from functional requirements.
    Responsible for processing the presentation tier and applying the relevant business logic to it.
    \hrule
    Data tier, this tier is responsible for storage and persistence of user and business-specific data.
    Tier may include files on a local file system or a database.

    \paragraph{Why so popular?}
    Fits a large number of use cases.
    A lot of web based services fit this model, shops, news sites etc etc.
    Very easy to scale horizontally to take large traffic volumes and handle more data.

    \paragraph{How does it scale}
    The presentation tier runs on a users device so it scales by itself....get a better phone / computer etc.
    \hrule
    Application Tier - if it is kept stateless, application instances can be kept behind a load balancer and run as many instances as we need.
    \hrule
    Database / Persistence tier - can be easily scaled if a well established distributed data base is used. DB scala - replication, partioning, sharding etc.

    \hrule
    This architectural pattern is very easy to maintain and develop because all the logic is concentrated in one place; the application tier.
    Most backend development should happen in the application tier.

    No need to worry about integration of different code bases, services or projects.

    \paragraph{Weaknesses of this pattern}

    Major drawback, the monolithic structure of the logic tier.
    Business logic should not be placed in the presentation or data tier.
    Issue is all the business logic is concentrated in a single user code base that runs as a single runtime unit.
    The implications of this, each instance of the application becomes too CPU intensive and will consume too much memory.
    This makes the application too slow and less responsive.
    This can especially be a problem with memory managed/ GC language like Java and C\# (maybe add go to this list)
    As a result applications can have longer and more frequent garbage collections.
    This can lead to requiring an upgrade of the computer the application is running on.
    Vertical scaling is both expensive and limited.

    \paragraph{}
    The second impact of the monolithic application tier is low development velocity.
    More complex code base makes it harder to develop maintian and reason about the code.
    Hiring more developers to solve this problem will not solve the problem/ add value.
    Why?
    Because more concurrent developers will simply cause more merge conflicts and higher overhead.

    \paragraph{}
    Mitigation: Could split the code into modules based on logic.
    The modules will be somewhat tightly coupled since we can release new versions of those modules only when the applications has been upgraded.
    I.e the organisational scalability of the Three-tier architecture is limited.
    Best when the code base is relatively small and not that complicated and will be maintained by a small team of developers.

    \paragraph{Examples}
    Early startups, well established companies that fit the criteria above.

    \paragraph{Variations of the Multi-Tier Architecture}
    One Tier
    Two tier - Business logic and presentation tier is combined.
    Think mobile or desktop application.
    Data tier handles persistence.

    \hrule
    More complexity - four tier.
    Between the presentation tier and the business logic tier.
    Between this tier functionality that does not belong in either can be seprated.
    Could introduce API Gateway tier that handles security,caching, data formats etc when communicating between different systems.

    \hrule
    More than four tiers is rare since they do not usually provide more value and simply adds more performance overhead.
    The source of this overhead comes from not being able to bypass tiers so as to avoid tight coupling.
    Every request would have to pass through multiple services which can increase response time/latency.
    Rarely a request would have to do so much travelling.

    \subsection{Microservices}

    \paragraph{Motivations}
    The size and complexity of the code base has grown, troubleshooting and adding new features.
    building, testing and even loading code into the IDE is now cumbersome.
    Organizationally, we now have problems.
    More developers leads to more merge conflicts, longer, larger and less productive meetings.
    At this point microservices should be considered.

    \paragraph{Microservices}
    Organises business logic as collection of loosely coupled and independently deployed services.
    Each service is owned by a small team and has a narrow scope of responsibility.

    \paragraph{Advantages}
    Codebases are now \textit{smaller} .
    Development is a lot easier and faster along with deployment and testing.
    Simply because there are fewer things.
    Code is easier to reason about and features can be added faster.
    New developers can become productive faster.
    Regarding performance and scalability, each microservice becomes less CPU intensive and takes less memory and can now run much more smoothly on commodity hardware (cloud).
    Can scala horizontally by adding more instances of low-end computers.
    Organizationally, the advantages are each service can be independently developed, maintained and deployed by a separate small team.
    This can lead to high throughput from the organisation as a whole.
    Each team can be autonomous with regard to tech stack, release schedule or process they want to follow.
    Better security in the form of fault isolation, i.e if one service starts crashing its easier to isolate and solve.

    \paragraph{Note} - Organisations can move to this architecture too quickly, without considering two factors.
    First - theoretically can achieve all those benefits from migrating to microservices, but they don't just happen and this can easily end up as a big ball of mud
    Second - microservices come with a fair amount of overhead and challenges.

    \paragraph{Pre microservices}
    To achieve full organisational decoupling so that each team can operate independently, services need to be logically seprated in way that every changed in the system can happen only in one service.
    This is to avoid involving multiple teams.
    If a change requires multiple teams then not much is gained from this migration.

    \paragraph{Microservices best practices}
    Single responsibility principle - each microservice needs to be responsible for one business cpabilyt domain resource or action.
    Monolithic API gateway can be decomposed into multiple microservices, making them more lightweight and specialised.
    Second practice is to make sure there is no coupling between different services, and this can be achieved by having a differnt database for each service.
    Note: If two services share a database then every single schema or document structure change will result in complext coordination between teams.
    If each service has its own DB then it just becomes an implementation detail that can be easily updated or replaced completly without implacting the rest of the system.

    \paragraph{Splitting the data}
    When splitting a monolithic database the data has to be split in a way that each microservice can be completely independent and fully capable of doing its work while minimising the need to call other services.
    Data duplication is expected and is normal in this scenario.
    Duplication is a tradeoff here and an overhead of using this architecture.

    Following best practices will help achieve a positive outcome when moving to microservices.

    \paragraph{Practical note}
    Microservices brings in additional complexity and overhead.
    This only brings business value when the system reaches a certain size and the organisation is of a certain scale.
    Start with a monolith approach first.
    When that is no longer working for the use case, then move to micro services.

    \paragraph{Conclusion}

    The benefits of microservices architecture.
    Higher organisational and operation scalability, better performance, faster development, better security.
    Best practices are required to realise these advantages and are only really applicable to systems and organisations of a certain size.

    \subsection{Event-Driven Architecture}

    \subsubsection{Motivation}
    If microservice A want to communicate with Microservice B, then not only does it require awareness of service B, it also needs to know how to call what API microservice B provides and how to call it at run time.
    Also, microservice A has to call microservice B synchronously and wait for its response (No always async etc????).
    Microservice A has a dependency on microservice B\@.

    \paragraph
    In an event driven architecture, instead of direct messages that issue commands or request that ask for data, we have only events.
    An event is an immutable statement of a fact or a change.
    In an Event-Driven architecture we have three components.
    Sending side - We have event emitters, which are also referred to as producers.
    Receiving side - We have event consumers
    And inbetween we have the event channel (a message broker), Kafka?

    \paragraph{Advantages}
    When we use the Event-Driven Architecture style with microservices we can get a lot of benefits.
    Now the dependency between Microservice A and B is removed.
    Microservice A doesn't need to know anything about the existence of microservice B
    And once microservice A produces the event, it doesn't need to wait for any response from any consumer.
    Because services don't need to know about each other's existence or API and all the messages are exchanged completely asynchronously, we can decouple microservices more effectively.
    This results in higher scalability.
    More services or integrations can be added to to the system without making any changes.
    Horizontal and organisational scalability is achieved via EDA.
    This architecture also allows us to analyse streams of data, detect patterns and act up on them in real time.

    \paragraph{Further advantages}
    When all events that occur in a system are stored in a message broker, in addition to data analysis an very powerful architecture pattern can be implemented.
    This is called event sourcing.
    By using event sourcing events can be replayed to identify the current state of a system rather than saving the state in a database.
    Because events are immutable we're never modifying them.
    We simply append new events to the log as they come.
    Using event sourcing we can store events for as long as we want.
    Querying can be made faster by adding snapshot events.

    \paragraph{CQRS}
    Another architectural pattern that can be implemented as a result of EDA
    Command query responsibility segregation.
    This solves two problems.

    First problem is optimizing a database that has a high read and update operations.
    In this scenario concurrent operations to the same record or tables contend with each other making the system slow.
    Additionally, if we use a distributed database, generally we can optimize it for one type of operation at the expense of teh other.
    (Cap theorem?)
    Ad-hoc solution - We can optimize one operation at the expense of the other when an application is read/write heavy.
    However, when the both operations are equally important we have a problem.
    CQRS architectural pattern allows us to separate Update and Read operations into separate databases, sitting behind separate services.
    In this case service A would take all the update operations and perform them in their own database, where it optimally stores the data for such updates.
    Additionally, every time an update operation is performed if publishes an event into a message broker.
    Meanwhile, service B will subscribe to those update events and applies those changes in its own read optimised database and now all read operations will go to service B.
    Both update and read operations can go to separate services without any contention/interference.

    Further, the data can be optimized for each type of operation.

    The second problem that CQRS helps solve is joining multiple tables that are located in separate databases that belong to two different microservices.
    Prior to monoliths all data was in one table.
    If that was a relational database all the records could be joined and then analysed.
    Post microservices, this is not the case assuming that best practice has been followed.
    Joins are harder now.
    Requests need to be sent to each service separately which is slower.
    This data needs to be combined programmatically because now we potentially have different types of databases.
    Some of these databases may not even be relational.

    CQRS solves this problem.
    Every time there is a change of data in a services database, those services would publish those changes as an event to which other services subscribe to.
    The other service will store what is called a materialized view of the joined read-to-query data from both service A and service B in its own read-only database.
    Whenever we need to get a join view we can just send a request to the joining service rather than send a request to two services.

    \paragraph{summary of EDA}
    When combined with microservices this allows for decoupling of services potentially allowing for horizontal and organization scalability.
    Event driven architecture allows us to analyse and respond to large streams of data in real time.

    \paragraph{Pattern within EDA}
    Event sourcing - allows for the auditing and storage of the current state of a business entity by only appending events and replaying them when needed.
    CQRS - Allows for database optimisation for both updates and reads by splitting the operations into separate services.
    This allows for the efficient joining of data from separate services.

    \paragraph{Note} EDA and Microservices aren't a requirement for each other but they are commonly used together to achieve greater decoupling.
    Issues to address with Microservices, organisational scaling and technical problems relating to scaling.
    Refactoring is harder and application becomes less stable, small issues can jeopardise the whole system.
    Solution, organise business logic as loosely coupled deployable independently deployable services.
    Each service is owned by a small team and has narrow scope of responsibility.
    Now we have organisational scalability, building is faster, less burden on the developer, each binary is smaller in size.
    Testing and reasoning is easier, onboarding is faster and this development velocity is faster.
    Hardware demands are easier since commodity hardware can now be used due to less CPU and memory usage.
    Flexible tool choice for development teams.
    Refactoring is easier.
    Higher system stability, the damage caused by a bug etc is much smaller since each service is deployed separately.

    \paragraph{Barriers to implementing microservices}
    Method calls have now become network calls between different computers bringing in latency issues.
    In distributed systems each component is unreliable.
    Although testing is faster no guarantee all the services will work when deployed together.
    This can lead to complicated integration tests than can impact productivity.
    Would be hard to understand which team owns the integration tests.
    Fixing bugs and troubleshooting performance is much harder in microservices
    Incorrect scope identification for a service can also cause additional organizational overhead.
    Could lead to duplicated effort.


    \chapter{Microservices in depth}


    \section{Migration to microservices architecture}


    \section{Principles and Best Practices}


    \section{Event driven architecture}


    \section{Event driven microservices}


    \section{Testing Microservices and Event-Driven Architecture}

    \paragraph{Motivation} Before deploying a system we to production we need to gain confidence through automated tests.

    \subsection{Monoliths}
    TODO get testing pyramid picture
    Three categories - Unit, Integration and End to End tests

    UNit tests - cheapest to maintain, they are small, easy to write and fast to execute because they are so cheap.
    Include a high number of them.
    Located at the bottom of the pyramid.
    Provide the least confidence about the overall system since they only test each unit in isolation.
    Once we run the application we have no idea if those units will work together or not

    Integration tests
    Those tests verify that different units and systems we integrate with such as database or message broker actually work together.
    Those tests are bigger and slower so we should have fewer of the than unit tests.
    After running them we have more confidence in our tests

    End to end tests-- at the top
    These tests run the entire system including UI and database and verify they work as expected
    From an end user perspective each such test should represent a particular user journey or business requirement and ensure it matches the specification for the application
    Heaviest and most expensive tests to run.
    Minimise the amount of these tests.
    These tests provide the highest confidence that our system will work as intended in production

    \subsection{Translating testing pyramid into microservices}
    Each team should follow the same steps as the monolith, i.e pyramid for each microservice
    Then we treat each microservice as a small unit that is part of the larger system and put it in a larger testing pyramid
    So just like in case of unit tests testing each microservice in isolation is essential but not enough to increase the confidence that all the microservice will work together
    Need another layer of integration tests.
    Those integration tests verify that every pair of microservices can talk to each other using the agreed API while mocking the rest of our system. Hmmmmmmm.
    To complete this pyramid we need to add system level end to end tests at the very top
    Those tests in theory should run all out microservices, databases, message brokers and frontends in a test environment.
    This should verify that all components work together as expected.

    \subsection{Challenges}
    First challenge - end to end tests are hard to setup and maintain despite providing the most confidence
    It unclear what team should own this environment and when one of the microservice teams breaks their build the entire test pipeline will be broken.
    This can result in teams being blocked and unable to make any releases to production.
    Alternatively, developers may just start ignoring end to end tests lol and release their microservices anyway which makes tests a liability with no benefit.
    Very costly to run what is essentially a duplicate environment of the production environment even if the scale is smaller.
    - In practice some companies spend disproportionately too much effort building and maintaining those few tests while other companies decide to take the risk and simply don't bother investing in those tests at all.
    The second challenge of hte new pyramid is that even running integration tests can be quite difficult and creates a point of tight coupling between teams.
    --- When the that owns MMicroservice A that consumes the API of microservice B wants to run integration tests it needs to build, configure and run both microservice while Team A knows how to build and run their own microservice.
    -- It may be unclear to them how to setup microservice B, its even more difficult if microservice has many dependencies like a database or another service that needs to runnd or mocked.
    -- The team that owns Microservice B has a problem, this microservice may have multiple microserivces that consume its API.
    ---- To ensure that the changes that make in their API did not break all those API consumers they need to build, configure and run all those consumers to execute those tests.
    ---- This can easily get out of hand and slow down the development of our entire organisation.

    Third challenge is when using event driven architecture to decouple microservices.
    - in this case we have a microservice that produces events to a message broker and it actually doesnt always know which microservice consume those events.
    - This kind of decoupling is one of the benefits we want from event driven architecture
    - However now we can't really run any integration tests with those microservices
    - Also have the same problem when we are a team that owns a microservice, that consumes events from other microservices.
    --- Here again we, need to run those microservices and a message broker just to test that our microservice is able to consume those events and that the events did not change without out knowledge.

    \subsection{Contract Tests and Production Testing}
    Many companies invest disproportionally high efforts into end to end tests or abandon them altogether.
    Other challenges between the integration of microservices, which included their complexity and tight coupling between teams.

    \subsubsection{Testing challenges solutions}
    Integration tests
    - To deal with the complexity of integration tests is to use light weight mocking.
    - Mock the API layer of the microservice dependencies which we wish to integrate with and send it back a hard coded response if it receives the expected request.
    - Similarly other microservices can run mock consumers rather than running real microservices.
    -- They can write tests that make those mock consumers, send us requests and test that out microservice returns the expected service.
    -- The strategy reduces the coupling between the different team because if one team breaks their build the other team can continue running its tests.
    --- It also reduces the overhead of running real instances of the other microservices.
    ---> This strategy on its own has one major issue
    ----- The issue is that the contract between an API consumer and the API provider can get out of sync without those teams every detecting \textbf{it}
    -------> example - the API provider team may change its API update, its mock consumer and its tests and all their tests will pass successfully
    -------> But the communication about those changes may be lost or misunderstood by microservice A which consumes that API
    ---------> But the communication about those changes may be lost or misunderstood by microservice A which consumes that API. So that they either make incorrect changes and their tests will also pass.
    ----------> Once in production those microservices won't be able to communicate with each other and cause an outage.
    -----------> Motivates the existence of ---contract tests---
    -----------> Contract tests work by using a dedicated tool to keep the mock API provider and the mock API consumer in sync through a shared contract
    -----------> When the API consumer team runs their tests, they are run against their mock API provider is recoreded along with the expected response into a contract file
    -----------> This contract file is the shared with the team that owns Microservice B, which provides that API using this contract
    -----------> The team that owns Microservice B replays all those recorded requests to the real microservice B and verifies that the responses it gets are the same as recorded in the contract
    -----------> And if microservice B has many consumers, it will take all their recorded contracts, create those mock consumers and run each one against their actual APi implementation
    End Goal of this ----- each team can run its own integration tests without dealing with the complexities of building, configuring and running other microservices.
    ---- the contract test tools guarantee that each team tests against the most up to date and correct contract shared between those microservices
    ----> Can also extend the idea of contract tests to integration tests between microservices. ****Insert concrete example****
    ----> Contract tests can simplify running integration tests for microservices that communicate synchronously and asynchronously but still give us high confidence that when we deploy those microservieces to prodoction, they can comunicate with each other as we can expect.

    \subsection{End to end tests}
    Contract tests can be a substitute for integration tests, they are not a substitute for ene to end tests
    -- If setting up end to end tests is is not feasible the alternative is testing in production
    -- One way is using a gradual release using blue green deployment in combination with canary testing
    --- A blue green deployment is a safe way to release a new microservice version to production using two identical production using two identical production using two identical production environments withouth any downtime durign release.
    -- Blue environment is a set of servers or containers that run out old version, and the green environment is a set of servers or containers that run the new version we want to release
    -- Once we deploy the new version to the green environment, no real user traffic is going to it.
    -- This is an opportunity to increase our confidence by running automated and even manual tests against those servers without impacting real users.
    -- After we run those tests we can shift a portion of the production traffic to the green environment and monitor the new version for performance and functional issues.
    ------> This process is called canary testing
    ------> If we detect an issue, we immediately direct the traffic back from the green environment to the blue environment with minimal impact on users
    ------> On the other hand, if no issues are detected we direct all the production traffic from the blue environment to the green environment and gradually decommission the blue environment since it's no longer needed/

    \subsection{Summary}

    These alternatives should be used only if they are setting up ****real**** microservices in development stage for testing purposes is too complex or expensive.


    \section{Observability in Microservices Architecture}

    \subsection{What is observabiilty}

    \paragraph{Three pillars}
%
%    \begin{itemize}
%        \item Distributed Logging
%        \item Metrics
%        \item Distributed Tracing
%    \end{itemize}


    Monitoring is the process of collecting and analysing and displaying a predefined set of metrics and by attaching those metrics to alerts.
    Via alerts we can find out if something has gone wrong.
    Monitoring tools and dashboards will only tell us when stuff goes wrong but they will not tell us what the problem is.

    Observability allows for active debugging and searching for patterns, follow inputs and outputs, and get insights into the behaviour of our system.
    This allows us to follow the flow of individual transactions or events across the entire system.
    Can discover performance bottlenecks and find the source of the problem.

    Monitoring is important for any system.
    Observability is primarily critical for microservices architecture.
    Monoliths are easy to debug since it is effectively one application.
    Worst case scenario.
    Can ssh into a monolith and inspect its logs and get an idea of what part of the code is causing the performance issues or is throwing the exception.

    On the other hand, in microservices architecture, a single user request may involve several microservices and databases.
    These can communicate through a message broker like kafka.
    If an issue happens at some point in the transaction, it can be difficult to find out which service is happened in.
    Its is further challenging as all those computers run as a group of instances on different computers thus making it harder.
    Many issues occur at the boundaries of microservices rather than in the microservice code itself.
    Being able to trace the path of requests through a set of microservices is very important.

    Typically, having just one type of ``single'' is no sufficient to debug microservices or and EDA system.

    \subsection{Signals}
    When referring to signals three types of signal are referred to.
    Know as the three pillars of observability.
    One of those pillars are distributed logging metrics, distributed tracing logs and append only files that record individual events happening within an application process.
    Those events are accompanied by metadata such as the timestamp of the event, the reqeust that triggered it and the method class or application where the event happened and so on.

    Metrics are regularly sampled data points represented as numeric values such as counters, distributions or gauges.

    \subsubsection{Metric examples}
    Examples include counters of the number of requests per minute, errors per hour, distributions of latencies or gauges that represent the current CPU or memory usage.

    \subsubsection{Tracing}
    Traces represent the path a given request takes throughout several microservices and the time each microservice takes to process the request.
    Traces may include additional information such as response headers, response codes and so on.

    When receive an alert about an issue or manually detected issue using dashboards we can further debug the issue using a combination of those signals.
    Can trace individual requests, isolate the issue to a particular microservice or API and even down to an individual method even line of code that causes the bug of performance bottleneck.

    Using those signals, we can also get enough insight into how to solve that issue.
    Typical solutions can include a rollback, hotfix or changes to the infrastructure.
    Infrastructure changes can include things like adding more service instances, diverting, traffic to other regions, data centres....etc.

    \subsection{Distributed Logging}

    Logging is a basic way to provide insights into the current state of an application.
    Each log line can represent an applications even like receiving a new request or an action like performing a database query or starting a complex processing operation.
    It's also a way to record exceptions and errors in a method accompanied by the set of parameters that led to that issue.
    This information is useful to debug and fix the issue and fix bugs.

    Should be noted that in a microservices architecture we can have thousands of instances of different microservices producing millions of lines per day.
    Consider the practicality.

    A solution to this collection of large and very useful data is to collect them into a centralised and highly scalable logging system.
    The system needs to parse and index those logs so they can be easily searched by patterns of text and grouped and filtered by attributes like host, microservice, time period or region to make searching easy.

    Good practice to follow a predefined structure or schema and the same terminology across different events within a microservice and across separate microservices.

    Log lines should be easily readable by humans, but also easily readable by machines so they can be efficiently,parsed, grouped and analysed.

    Can be extremely important when we're faced with a time sensitive issue affecting users and we need to find a root cause solution quickly.

    Log structures include log FMT (key value pairs), Json and XML.

    Next best practice is to assign a log level or severity to each log line, depending on the framework or system we use.

    Typical log levels are trace, debug, infor, warn, error and fatal.
    Adds info that logs can be sliced on in order to reduce noise and alert fatigue.
    This is useful for the on call engineer.
    Can use automated tools to search and group events on those levels of severity.
    Those tools can alert us or create a ticket automatically for someone to work on.

    Events that indicate potential problems such as high processing time of outgoing or receiving requests containing unexpected values must be logged at the warn level

    This way we can filter on the log level and potentially prevent future issues by addressing those events.

    A developer can look at all log events at a fine level to see details on the debug and trace levels.

    \paragraph{Next best practice}
    The next best practice is to use a unique correlation ID for every user request or transaction and adding this ID for each corresponding to an event or step and processing that request and transaction.
    Since each microservice instance is likely processing multiple user requests concurrently.
    This helps search and filter only the events related to teh request in question.
    Can also help us to see the sequence of events for a given request across multiple microservices.

    \paragraph{Next best practice}
    Provide as much contextual information to each log line as possible
    - Want to include the parameters that led to this error.
    - If we have a database query that took a very long time to complete, logging the exact query and the content etc can help fix/mititgate.

    Common data we want to include in every log line is
    - Name of the service
    - Th emitted log event
    - The host name where that happened
    - the userID or some other identifier that can add more context to who initiated the operation and of course the timestamp that that event happened

    Two considerations in mind
    - First we should only log information that is critical for debugging because on a large scala system storage and processing can be very expesnive
    - Second we should never log sensitive or personally identifiable information such as usernames, passwords, security numbers, emails, credit card numbers and so on.
    - These details are sometimes helpful but can cause a huge legal risk for the company in the event of a security breach
    - Add extra complexity in relation to security and data retention, compliance and generally is unethical
    --- Idea of a random engineer with access to personal information is not a good one

    \subsection{Metrics}
    A measurable or accountable signals of software that helps us monitor performance of a system and detect anomalies when they occur.
    Usually come in numerical values so we can easily quantify them and set alerts based on their direct or derived values because they are just numbers.

    Out of the three pillars of observability they are best collected, visualised and organised into dashboards.

    These production dashboards are a critical tool for us when a production issue needs our attention instead of having to search and read though hundreds of log lines, we can look at two.

    Questions to ask ask as a team about microservices.
    - Should ask what should we measure
    - What should we collect
    - what should we monitor
    - as well as why we can't collect all of them

    Somthing to consider is that the number of signals we can collect is pretty large on the resource levle.
    We can measure many things but they are not helpful in certain situations.
    Additionally we can instrument our application and measure anything starting from the number of requests we recieve per minute to the number of time a critical piece of logic is being executed.
    However collecting anything and everything that can be measured is a big anti pattern.

    Storing so many signals from each server can be expensive.
    Talking about a large scale system with 10s of 100s of microservices.
    Even if cost is not an issue the next problem is information overload.
    -- Consider on call engineer hahahhaahaha
    -- What metrics should this person look at, so many metrics they won't fit on the screen
    --- Makes it hard to find the anomalies
    --- even if do notice the anomalies it is hard to know which metric is the symptom and which is the cause
    --- Instead be smart about metrics.
    --- Can leverage decades of experience from companies which have already been doing distributed systems and focus on the five most types of signals which will give us the most infromation and the least amount of noise.

    \paragraph{Five types of signals that will give us the most knowledge}
    The first is Google Searches for Gold and Signals, which focuses on user facing metrics
    The second is the use method (Brandon Gregg) which focuses more on system resources.
    By combining these categories of singlals we have five categoies of signals whicih are
    - Traffic errors
    - Latency
    - Saturation
    - Utilization

    Traffic is the amount of demand being placed on our system per unit of time.
    Number of HTTP request per second or minute that recevies Http traffic.
    We can also measure the number of queries or transactions per second or per minute on the database or message broker.
    Can measure the number of events it receives adn the number of events it delivers to consumers.
    --- In some cases a single request to a microservice results in many out going requests and incoming reqeusts separately.
    --- That's because open connections to other services consume system resources and can directly impact the microservice performance.

    Next category of metrics is errors.
    - When we measure errors, we are interested in the error rate and the type of errors we are getting if possible.
    - A good signal to setup an alert for because the error rate goes up.
    At this point users have already been impacted.
    - In the case of an increase in the number of exceptions in out application, we may not be able to show the error type as a number value, so we defer this information to logging.
    -- However if we start receiving a HTTP response status code that is different from 200 from a service we depend on, we can use that as a metric which will be very helpfulin troubleshooting production issues.
    -- Also latency sensitive system swe can count successful responses as errors if they exceed so latency threshold that we set ahead of time for an even driven microservice
    -- Can also measure the rate of events that it fails to process and the reason for the failure, if its possible to classify in a message broker, we can measure error singals like the number of events it failed to deliver to customer.

    On the database side, we can measure the number of aborted transactions, disk failures and so on.

    Next type of signal is latency
    -- The time it takes for a service to process a request.
    -- Seems straight forward but there are a few thing to consider to measure it correctly.
    -- The first important consideration is to not just look at the average latency, but always consider the full latency distribution....especially tail latency.
    ---- This could help identify performance bottlenecks
    -- Another consideration is the separation the latency of successful operations from the failed operations.
    -- If we mix failed and successful operations we may get the wrong data or a misleading average

    Next type is saturation
    - Measures how overloaded a full service or a given resource is.
    - Important for a service that has a queue, whether its an external queue like a message broker, an microservice internal queue or CPU
    --- Too many things in a queue means the system cannot keep up with the demand at present - this would indicate a scalability issue in the respective area.... CPU, message broker,database etc
    --- If work in a microservice keeps growing it could mean that part of our microservice is too slow and this instance may crash too soon with an out-of-memory exception.
    --- Finally having visibility into saturation can also explain why out users have long latency or requests from other services that are timing out.

    Final signal is utilisation
    -- how busy a resource is over a period of time
    -- Typically applies to resources with limited capacity like CPU, memory, disk space and so on.
    -- Its important to point out that in most cases we will see performance degradation before 100\% resource utilisation
    ---- Important to set critical alerts before we reach that critical level
    ------- CPU getting close to over utilisation, we need to scale out and add more service instances other wise we may get higher latency issues etc
    ---- Similarly if we see out database is getting close to running out of storage, we need to add more database instances to keep up with the growing amounts of data
    We want to measure utilisation with high granularity, not just an average over minutes otherwise we will miss the short periods of utilisation that may be attributed to performance bottlenecks or other ineffciencies in out processing

    These arent the only five signals that should be collected
    --- Consider business use case/scenario
    --- Additional can be collected for observability depending on logic and the specifics of out microservice

    Those five types of signals are the most common that apply to any system and give us the most value by tracking them.

    \subsection{Distributed Tracing}
    Distributed tracing is a method of tracking requests as they flow through the entire systems.
    Starts at the clients device all the through the backend services and databases.
    As the request is being traced we collect critical performance information about the time each part of the system is processing it
    Allows engineers to visualise the entire flow and understand all the components involved in processing the reqeust and the time it took for each component to do its work.
    Extremely valuable for troubleshooting bugs or errors tha lead to wrong behaviour or performance bottlenecks.
    Usually distributed tracing is not enough to tell is what is happening exactly but it is enough to narrow down out search to a particular component or a communication problem between two componenets.
    After we know where the problem is happening we can use other observability pillars (logs, and metrics) to debug further.

    \paragraph{How distributed tracing works}
    When the initial reqeust is being made we generate a unique trace ID and place it into an object called a trace context
    This trace context object contains key data about the entire trace as the request flow through the services.
    That context is propagated through Http headers or message headers inside events.
    Just passing the tracing context from service to service is not enough for the application instances to collect the tracing data.
    Need to instrument them using a tracing instrumentation library or SDK

    Tracing libraries come in different languages so even in a polyglot system we can get a complete trace.

    Now as soon as the service instance receives the trace context, it collects the necessary data and propagates the context for the next service

    At the eng od the transaction, each service instance that was involved has its own measurements and data which can later aggregated by the trace ID to visualise all parts of the transaction and how long each part of the transaction took.

    Trace is broken into logical units of work which are called spans.
    Trace spans can be coarse grained like the processing of a request by a service or a query by a database

    They can be manually created by the developers using the instrumentation library.
    So different units of work within a service can be visualized and measured separately.

    This way if one logical part seems slower than usual we can investigate it as a potential performance bottleneck.

    If an expected span is missing we may have a bug we need to debug further to increase the granularity even futher.
    Can connect related pieces of work together by organising spans in a hierarchy with a parent child relationship.

    \paragraph{Distributed tracing data collection}
    General approach.
    Once the trace data is collected inside each service instance, it is pulled by an agent which runs as a separate process on the same host with each service instance.
    Those agents then send the tracing data to a central queue or topic in a message broker.
    The all the tracing data that comes from many services is analysed, aggregated and stored in a databse by a big data processor.
    Later a developer can query and visualise this data using a tracing UI in the browser.

    \paragraph{Challenges}
    Need to manually introduce code to collect the data for these systems.
    Usually requires dependency on a certain library and to learn how to use it correctly.
    If this is not done properly out spans may be too broad, lacking granularity or important data.
    Can go as for making the traces useless.

    The second challenge is cost
    - Need to run an agent on each microservice host which consumes its own CPU and memory
    - Then the collected data must be sent over the network which requires additional bandwidth
    - The we need to run a big data pipeline with it own infrastructure to process those tracing logs that come from different services.
    --- This comes with its own cost
    - Biggest cost is storing those traces in a database and retaining them for at least for a few weeks.
    - This is so that developer can find them if they need to debug an issue
    - Considering scala most companies use sampling on the client side which may mean we get one trace out of 10,000 or 1000 requests.
    -- this can lower our storage costs but with such a high sampling ratio sometimes it makes it difficult to find a trace we are trying to debug.
    -- Another problem is the size of the trace and the amount of information contained in them.
    -- Typical Microservice or EDA deployment involves so many components that it is difficult for even a human to read it.

    Despite the challenges distributed tracing is very useful when debugging microservices.
    Help developer have the confidence to debug issues in production, find the root cause.

%    TODO add more details maybe????

    \subsection{Distributed Tracing Solutions}
    \begin{itemize}
        \item OpenTelemetry
        \item Jaeger
        \item Zipkin
        \item Uptrace
    \end{itemize}


    \section{Deployment of Microservices and Event-Driven Architecture in Production}

    \subsection{Microservices Deployments - Cloud Virtual Machine, Dedicated Hosts and Instances}
    First and most common way to deploy microservices in the cloud environment is by using cloud virtual machines.
    ---Virtual machine blah blah blah - an isolated environment, running on top of a real physical computer.
    --- Virtual machines act like a virtual computer with its own operating system and virtual resources like CPU memory network interface storage
    --- Those virtual resources are allocated, managed and mapped to real resources on the physical hardware by another layer called the hypervisor
    --- By using virtual machines, the cloud providers can split each physical server into multiple VMs depends on how we configure it and how much we pay for that VM
    --- In turn this allows for cloud providers to provide competitive prices, which makes cloud VMs a very attractive option for running microservices
    ---- Main benefit of using cloud virtual machines for running microservices is the affordable and flexible pricing
    ---- Typical pricing model is pay per use -- we pay only for the cloud VMs that we're renting them
    ----> and the rate for each VM depends on the CPU memory and network bandwidth capabilities that we request
    ----> Downside of this deployment is security risks due to multi tenancy
    ----> Theoretically, when we run two VMs on the same server, those two VMs on the same server, those two VMs are completely isolated from each other
    -----> If those two Vms are running two database instances and each instance belongs to completely different organisation, a security breach in one should not pose a risk to the other VM
    -------> Need to consider that the hypervisor is a software that was written by human beings and all security configurations are managed and updated by human beings
    ----------> So in practice it is possible for a hacker to gain access to a VM that was poorly secured by the other software engineers / organisation
    -------> And because the cloud vendor allocated out VM to run on the same host, that hacker may manage to hack into out system and steal our data
    ---> Cloud vendors and the companies that own the hypervisor make great efforts to prevent this from happening,
    -----> The probability of this happening is very small but it is still there
    --------> Due to compliance issues in certain industries, we may be unable to tolerate even that small risk.
    Example businesses: Banking, healthcare, government and national security related services.
    Second issue of running microservices in multi-tenant environments is potentially lower performance due to a \textit{Noisy Neighbor}

    Theoretically, the hypervisor shoudl be able to completly isolate and allocate each hardware resource to each VM as configured ahead of time.
    However, in practice, not all resources can be accurately allocartd.

    \paragraph{Example}
    A CPU has 16 cores, then the hypervisor could easily split those 16 cores evenly.
    However the network bandwidth of a network card or the access to an internal bus that transfers data to and from a storage device can't be easily rationed in an accurate way and many other physical resources.
    -- This is true for other aspects, this a physical server at the end of the day.
    -- Additionally, the hypervisor itself may consumer some CPU and memory for its own user.
    -- Not really a lot of data to suggest that running a very intensive workload on one VM can significantly impact the performance of the other VMs.
    -- In theory and in practice that impact is still there
    ----> For very latency sensitive systems like high frequency trading systems, gaming or video streaming multitenant deployments with cloud VMs is not the best option


    \section{Single tenant}
    Dedicated hosts or instances, ask cloud provider to run VMs on servers that are dedicated to the account that belongs to our VMs on servers that are dedicated to the account that belongs.
    This means that the only tenants will have on the same host are instances of our microservices databases for our organisation only.
    More expensive alternative if we are in an industry that doesn't allow us to share infrastructure
    --- They charge more because cloud providers can't allocate their hardware as efficiently as the multi tenant deployments
    ------ Some cloud providers, even allow us to rent or reserve and entire host just for our organisation
    -- This gives us direct access to the hosts hardware resources, which can eliminate the noisy neighbor effect and reduces the impact of virtualization by the hypervisor
    -- Main downside is that this is way more costly than multi tenant cloud VMs


    \section{Summary}
    - Multi tenant VM deployment provides us with the best pricing but doesn't provide us with the most optimal security or performance
    - If we need additional security, we can rent a single tenant dedicated instance which are a bit more expensive, but guarantee that onlyy VMs of our organisation can run on the same physical hardware
    - If we also need the most optimal performance and want to eliminate the possibility of a noise neighbor we can rent dedicated hosts,which is also the most expensive option.ó


    \paragraph{Multi-Tenant Virtual Machine Cloud Service}
    \begin{itemize}
        \item Amazon Elastic Computer Cloud (EC2) Instances
        \item GCP computer enginer
        \item Microsoft Azure Virtual Machines
    \end{itemize}

    \paragraph{Dedicated Hosts and Single Tenant Virtual Machine Cloud Services}
    \begin{itemize}
        \item Amazon Dedicated EC2 Instances
        \item Amazon EC2 Dedicated Hosts
        \item GCP Sole-Tenant Node
        \item Microsoft Azure Dedicated Hosts
    \end{itemize}

    \subsection{Serverless Deployment for Microservices using Function as a Service}

    \begin{itemize}
        \item AWS Lambda
        \item Cloud Functions
        \item Azure Functions
    \end{itemize}

    More event driven that other deployments.
    Used when the service is rarely used / low traffic most of the time.
    - If a cloud VM/ dedicated host is used we will be paying for rented hardware during the down time as well.
    - When the rarely used service is used there will be a traffic spike.
    - Need to configure and pay for a load balancer and maintian autoscalaing policies for this microservice, which adds more to the infrastructure running costs
    - Running a service for events that rarely occur is not very cost efficient
    - There is also the cost of maintaining it
    - The tream that owns the microservice in addition to eh business logic will also need to maintian a lot of boiler plate code that hanndles http reqeusts or events from another source that triggers that logic.
    - Also need to maintaing script to biuld, package and deploy our microservice binary with all its dependencies to a cloud server.
    - A lot of effort for events that rarely happen

    A solution is use a serverless offering called function as a service.
    - A cloud solution that allows for the architecting of a system in a fully even driven model, not only from the software perspective but also from the infrastrcutre perspective
    - Only need to provide cloud vendor with two things, the type of events we wanted to handle and the logic we wanted to executre when that event is triggered. Noting else
    - Only when the event is triggered will the cloud provider take out code, package it, deploy it to physical hardware and execute it
    - If the traffic follows a seasonal pattern where there are a lot of requests coming in a very short period of time, in that case the cloud provider will also handle the horizontal scalabiilty
    --- This allows for handling the traffic spike without the need to maintain any autoscaling policies of configuring a load balancer
    --- The pricing model is based on the number of requests our microservice receives as well as the memory, time to handle each request by running out logic.
    --- If a requests or event doesn't arrive we don't pay anything.
    -- CLear benefit is that a lot of money can be saved for events that happen very rarely.
    -- Can save a lot of money on infrastructure for seasonal workloads with rare but very high traffic spikes
    -- Cloud provider also handles all the operational scaling of that service
    -- Another benefit for both those types of workloads is that it allows us to save a lot of development cost and overhead as the cloud provider takes care of building, packaging and deploying out microservice.
    -- Trade offs
    -- If the traffic pattern changes the costs of the service may increase significantly
    -- If the business logic becomes more complex each time we receive a request or event handling it will require a lot more time and memory.
    -- As a result it will become more expensive for us to use this offering than deploying it on a cloud VM or even a dedicated host.
    -- Another downside is the performance of a serverless deployment is much less predictable than deploying business logic as part of a microservice we fully control.
    -- Serverless deployments like function as a service are not the best option for latency sensitive workloads.
    -- This is also less secure type of deployment because not only does our code run in a multi-tenant environment, we also expose our source code to the cloud provider.

    \paragraph{Summary}
    Function as a service for the correct workloads can be a very efficient way to deploy microservices.
    - If used incorrectly, or not for the right workload it can also be the most expensive options.
    - When it comes to security and performance this is also be the most expensive option.
    - When it comes to security and performance this is also the least optimal option of all the deployment types.

    \subsection{Containers for Microservices using Dev, Testing and Production}
    Problem to be solved -- Lack of parity between development environment, production environment.
    A.k.a - works on my machine but not in prod problem due to the differences between the configs in the two

    One solution is to develop and test the microservice in a production like Virtual Machine on our development computer.
    -- This leads to a new problem, the host operating system need to run another software layer called a hypervisor.
    -- Hypervisor runs just like any other program on our host operating system with its own kernel and operating system.
    -- This kernel manages all the guest operating system application processes, file system and security networking, memory and many other components.
    -- Also uses device drivers to interact with virtual hardware which is emulated by the hypervisor
    -- We can see that we have a lot of unnecessary overhead just to run a single microservice instance.
    -- If we want to run and test the integration of a few microservices, we need to run multiple virtual machines, each with its own operating system and it own kernel.
    ----- This creates a lot of overhead and makes everything slow and inefficient, which makes the development and testing very hard.
    -- Containers (docker etc) solve this problem by isolating what we want to osilate and sharing everything else.
    -- When we package our microservices into containers, each container image includes our microservice binary, the command to run it and all the dependencies it requires in complete isolation from any other microservice.
    -- When we create multiple instances of that image, each container has its own isolated file system, network interface and runtime.
    ---- **The OS kernel drivers and everything else that we don't need to isolate are shared among all the containers.
    ---- The overhead of running a few dozen of containers on our machine is minimal, making it attractive for developing and testing microservices
    -- Benefits of containers go beyond this, they are useful in a CI/CD pipeline which may use a different OS and hardware that development computers.
    -- This does not matter since container images are completely decoupled from the OS and the hardware so we can create them once and run them on any hardware and operating system that supports containers and container runtime.

    Disadvantages of using full VMs
    -- If two microservice instances run on the same cloud server each VM runs and entirely separate copy of the same operating system.
    ---- This duplication means we lose valuable memory CPU and storage resources to the operating system
    ---- Deploying and starting each VM can take minutes before being ready to accept traffic.
    ---- Another issue is cloud vendor lock in and lack of portability
    ---- When we create a VM image for a microservice,  the formate of this image may be cloud vendor specific, also the configuration describes the type of VM we want to rent is also cloud vendor specific.
    ---- If we get an attractive offer from another cloud vendor and we want to migrate completly, we will have to do a lot of work creating those new images and even this is an even bigger problem when we have a a multi cloud environment/ hybrid cloud environment.
    ---- Uses of hybrid cloud -- higher performance and greater security.
    ---- In these situations cloud VMs will be challenging to manage.
    ---- This is all solved by using containers.
    -- All we need to do is create those container images once and then deploy them on generic cloud VMs or even directly on dedicated hosts.
    -- In any environment the only requirement is to install the container runtime, which is the software that runs those containers and we're good to go.

    \subsection{Summary}
    Why containers
    -- Better portability between environments since we can build a microservice image once and then use it in development, QA, staging and production on any cloud provider hardware or operating system.
    -- Also get faster startup time because containers usually take a few milliseconds to deploy and run
    -- We can also save a lot of money on infrastructure because the hardware utilisation is much better when we use containers.
    --- Instead of renting multiple smaller VMs where we lose part of the CPU and memory to the operating system, we can rent larger VMs or even dedicated hosts
    --- When we deploy the containers of different microservices on a single VM or host we can better utilise the hardware because they share the same operating system kernel.
    ----- In many cases this means that we can run more intensive microservice instances for the same amount of hardware than when we use cloud VMs.

    We now have another problem to solve before being able to take advantage of containers in production
    -- The problem is that we have two layers of abstractions that need to be glued together
    -- We have cloud infrastructure abstraction, VMs etc which need to be rented and autoscaled based on the traffic or load on out system as well as other cloud managed systems like databases, message brokers and distributed logging.
    --- Also have container abstraction, where there are numerous abstractions representing different microservices.
    --- Each microservice image needs to be deployed as a group of containers instances on that infrastructure.....
    --- Then we need a way to discover and connect all those microservice containers to each other and the managed services through the network.
    --- We also need to manage the scalability and availability of each group of the containers so we can add more instances if the traffic goes up and remove instances when the traffic goes down.
    --- Additionally we need to automatically replace containers that crash and a way to update all the containers of the same microservice when a new version of the microservice is released.
    --- Doing this manually for 100s/1000s of container instances potentially running in different cloud providers is an incredibly difficult task.

    \subsubsection{Summary}
    .....................

    \subsection{Container Orchestration and Kubernetes for Microservices Architecture}


    \chapter{Cloud Architecure patterns}


    \section{Scalabilty patterns}


    \section{Performance Patterns for Data Intensive Systems}


    \section{Software Extensibility Architecure Patterns}


    \section{Reliability, Error Handling and Recovery Software Architecture Patterns}


    \section{Deployment and Production Testing Patterns}


    \chapter{Big Data Architeciture}

    \paragraph{Terminology} - Data sets which are too big or too complex or are produced at too fast that exceed the capacity of a traditional application.
    Characteristics are typically volume, variety, rate/velocity.
    This data can be used to gain insights/conclusions via visualisation, querying or predictive analysis.
    Can also be used to find anomalies, analysing logs.


    \section{Batch}
    requires the storage of incoming data in a distributed database or a distributed files system.
    Data is never modified, it is only added to the end.
    Key principal is that the data is processed in batches or records on a fixed schedule, or a fixed number of records that we want to process.
    The schedule can be adapted for the users/clients needs.
    Each time a batch processing job runs in can process new data and produce and up-to-date view of all the current data.
    This can be stored in a well-structured and indexed database that can be queried to get insights.
    Could analyse recent data or the whole data set.

    This data is not processed in real time.

    Note: This data can be used to create a Machine learning model.

    Batch processing provides the user with high availability, no downtime for users the old data view is still available.

    Batch processing is more efficient vs processing each piece of data individually. Also higher tolerance towards human error, with regards to bad code/ deployment issues.
    Batch processing can be used to perform complex data analysis of large data sets.....

    \paragraph{Drawbacks}
    Long delay between data coming in and the result we get from the processing job.
    The view is not real time which can be an issue in some use cases (Trading, 24/7 systems etc).
    Forces users to wait a long time before they can act on the insights from the system.
    May not know that the data is not in real time.


    \section{Real time processing / Streaming}
    Each new event into the system is placed into a queue or a message broker, on the other end there is a processing job that processes each bit of data as it comes through.
    After processing the processing job updates the database that provides querying capabilities for real time visualisation and analysis.

    \paragraph{Pros}
    Can respond to data immediately

    \paragraph{Cons}
    Hard to do any complex analysis in real time as a result insight may be poor compared to to batch.
    Hard to data fusion in real time, at different time points or analysing historic data. Only limited to recent data for predictions.


    \section{Lambda architecture}
    May need processes of both strategies.
    Lambda architecture takes advantages of both.
    Aims to find the balance between fault tolerance and comprehensive analysis of the data from batch processing and the low latency that we get from real time.

    In this architecture the infrastructure is divided into three layers, batch, speed, and serving layer.

    Data that enters the system is dispatched into both the batch layer and the speed layer simultaneously.
    The purpose of the batch layer is to manage out data set and be the system of records.
    The data in our master set is immutable and new data is appended, never modified.
    This data is usually on a distributed files system, optimized for storing big files containing massive amounts of data.
    The second purpose of the batch layer is to pre computer our batch views.
    Every time we run out batch processing job, it processes all the data that we have in out master data set.
    Once the processing is cimplete it indexes and store the data in a read only database.
    Typically, this overrides the existing pre computed views that we created the previous time we ran the processing job.
    Note: The batch layer aims at perfect accuracy and operates on the entire data set.

    \paragraph{Speed Layer}
    Data is sent to this layer in parallel.
    Real time strategy is used here.
    All the data goes into a queue or a message broker and is then picked up as it arrives by the processing job.
    Processing job analyses the new even and adds the processed even to the real time view ready for querying.
    This layer compensates for the latency in the batch layer.
    Unlike the batch layer it only operates on the most recent data and doesn't attempt to provided a complete view or make any data corrections.

    This layer exists between the last time a batch was run and the present moment/recent data.

    \paragraph{Serving layer}
    The serving layers purpose is to respond to queries and merge the data from both the batch and speed layer and update the real time views.


    \chapter{Components}

    \subsection{Load balancing}

    \subsection{Message Brokers}

    \subsubsection{Kafka}

    \subsection{Design Patterns}

    \subsubsection{Scalability}

    \paragraph{Load Balancing}

    \paragraph{Pipes and filters}

    \paragraph{Scatter and Gather}

    \subsubsection{Microservices specific}

    \paragraph{Execution orchestrator}

    \paragraph{Choreography}

    \subsubsection{Extensibility}

    \paragraph{Sidecar}

    \paragraph{Adapters}

    \paragraph{Frontends for Backends}

    \subsubsection{Availablility, error handling and recovery}

    \paragraph{Rate Limiting}

    \paragraph{Retry Pattern}

    \paragraph{Circuit breaker}

    \paragraph{Dead Letter Queue}

    \subsection{Data intensive applications patterns}

    \subsubsection{Map Reduce}

    \subsubsection{Saga}

    \subsubsection{Transactional Outbox}

    \subsubsection{Materialized View Pattern}

    \subsubsection{CQRS}

    \subsubsection{Event Sourcing}


    \section{Detailed Design}

    \subsection{Caching}
    Redis

    \subsection{Queues}
    Kafka

    \subsection{Protocols}

    \subsubsection{TCP}

    \subsubsection{UDP}

    \subsection{Threads and Concurrency}

    \paragraph{Actor model and Akka}
    Java low level and higher level

    \paragraph{Functional}
    Scala.....

    \paragraph{Application level}
    Scaling via instances


    \chapter{Databases}

    \paragraph{NoSQL}
    MongoDB.....graph databases

    \paragraph{SQL}

    \subsection{Networks}


    \section{Performance and Scalability}

    \subsection{Testing}

    \paragraph{Gatling}

    \paragraph{Code performance}
    Hmmmm maybe JVM specific here


    \section{Distributed Systems}

    \subsection{Clusters}
    Kubernetes

    \subsection{Storage}


    \chapter{Performance}


    \chapter{Scalability}


    \chapter{Reliability}


    \chapter{Tech stacks}


    \chapter{Deployments}
    Docker, Jenkins


    \chapter{System design security concerns}


    \section{Introduction}
    Cover security from a system design point of view.


    \section{Network security}


    \section{Encryption}


    \section{Digital signatures}


    \section{Authentication}


    \chapter{Useful resources}
    Stuff used to create these notes

    https://www.udemy.com/course/software-architecture-design-of-modern-large-scale-systems
    https://www.udemy.com/course/the-complete-microservices-event-driven-architecture

    \newpage

% Example of including a code snippet


    \section{Code Example}
    \begin{lstlisting}[language=Java, caption=Java Code for a Simple Cache]
public class SimpleCache {
    private Map<String, String> cache = new HashMap<>();

    public String get(String key) {
        return cache.get(key);
    }

    public void put(String key, String value) {
        cache.put(key, value);
    }
}
    \end{lstlisting}


    \chapter{Diagrams}

    \begin{figure}[h]
        \centering
        \includegraphics[width=0.8\textwidth]{basic-architecture} % Change to your filename
        \caption{Three tier architecture}
        \label{fig:drawio-diagram}
    \end{figure}


    \chapter{Architecture examples}


\end{document}
