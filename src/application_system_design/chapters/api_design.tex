\chapter{API Design}

\section{Motivations} - After gathering requirements, the system can be thought of as a black box whose behaviour has a well-designed interface.
An interface is a contract between the engineers who implement the system and the client applications which uses the system.
Such an interface is a called an API since it is going to be called by other applications remotely through a network.
Not to be confused with a programming library.
Applications calling the Api could be front-end clients like mobiles and web browsers or they can be other backend systems including those from other companies.
After internal design each component will be called by other applications within our system.

\subsection{Categories of APIs}

There are generally three categories of APIs

\begin{itemize}
    \item Public APIs
    \item Private or Internal APIs
    \item Partner APIs
\end{itemize}

\subsection{Public APIs}
Exposed to the general public and any developer can use them and call them from their application.
Good general practice for public APIs is requiring the users to register before being allowed to send requests and use the system.
Allows for better control over who is using the system and how they are using it.
This in turn provides better security.
Can black list users who abuse the system.

\subsection{Private APIs}
Only exposed internally within the company.
Allow other teams or part of our organisation to take advantage of our system and provide value without exposing the system directly outside the organisation.

\subsection{Partner APIs}
Similar to public APIs but are only exposed to companies and users with a business relationship.
This can be in the form of buying a product or subscribing to a service.

\subsection{Benefits of APIs}
The benefits of APIs is that a client can enhance their business by using the system without knowing anything about the internal design or implementation.
Once an API is defined and exposed clients don't have to wait until the systems implementation has been finished.
Clients can start making immediate progress towards their integration goals.
Once exposed, it is easier to design and architect the internal structure or our system because the API effectively defines the endpoints/the different routes in the system the users can use.

\subsection{Desigining an API well}
Encapsulate the internal design and implementation and abstract it away from the users/developers that want to use the system.
Otherwise, it defeats the point of the API (I.e abstraction).
An API needs to be completely decoupled from internal design and implementation inorder to allow for design to change in the future; without breaking contract with the client.
The API needs to be easy to use and easy to understand, and impossible/hard to misuse on purpose.

\subsubsection{How to achieve this}
This can be achieved by having only one way to get certain data or perform a task rather than having many alternatives.
Having descriptive names for actions and resources, exposing only the information that the users need and no more than that.
Keep things consistent across our API will make using it a lot easier.
Another good practice is keeping operations idempotent as where possible.
I.e an operation that doesn't have any additional effect on the result, if it is performed more than once.
Idempotency is preferred since the API is going to be used over the network (i.e the network can be unreliable).
Messages can be lost or a critical component inside our system may go down and the message may or may not have been received.
The client or user will have no idea but can resend the request with no change of effect.
I.e same request will not be processed twice (i.e double transaction etc).

\subsubsection{Good Practice: Pagination}
Used when dealing with a large payload or dataset forming part of the reqeust.
Without it most clients will not be able to handle such a large payload or dataset and this would result in a poor user experience.
Pagination allows for the client to request only a small segment of the response by specifying the maximum size of each response from our system
and an offset within the overall dataset.
To receive the next segment increase the offset.

\subsubsection{Good Practice: Asynchronous APIs}
Another good practice for operations that take a long time to complete are asynchronous APIs.
The Client application receives a response immediately without having to wait for the final result.
The response usually includes some kind of identifier that allows the client application to track the progress
and status of an operation and eventually receive the final result.

\subsubsection{Very Important Practice: Explicit Versioning}
Explicitly versioning an API so that the client knows which API version they are currently using.
The motivation behind versioning APIs is to allow for non-backward compatible changes, thus allowing for two versions of "the same" API\@.
Meanwhile we can deprecate the old once gradually with good communication to the clients who are still using it.


\section{API Best Practices}
APIs can be defined in any way but some best practices have emerged.

\subsection{Remote Procedure Calls - RPC}
Ability to call a client application to execute a sub routine on a remote server.
Difference is that it looks like a remote method is being called in terms of the code the developer needs to write.
This feature is called local transparency.
Remote or local looks the same.
Some RPC frameworks allow for multiple programming languages so applications written in different languages can communicate.

\subsubsection{RPC: How it works}
API as well as the data types that are used in the API methods are developed using a special interface description language.
This is framework / implementation specific.
Effectively a schema or the communication between a remote client and a server in the system.
After this definition using the specialised language, a code generation tool or special compiler can be used to generate two separate implementations of the API tool.
This tool is framework specific.
One implementation is for the client, another is for the server.
Server side is called the server stub.
Client side is called client stub.
Stubs take care of all the implementation details of the remote procedure invocation.
All custom object types that we declare using the interface description language are compiled into classes or structs depending on the language.
Auto generate objects are referred to as Data Transfer Objects or DTOs.

Now at run time, whenever the client application calls that particular RPC method with some parameters the client stub takes care of coding the data.
This is called serialization or marshalling.
After serialisation, the connection to the remote server is initialised and the data is send over to the stub.

On the other end the stub is listening to the clients application messages and when the message is received it
unmarshalled/serialised and then the real implementation of the method is invoked on the server application.

Once the server is finished, the result is passed back through via marshalling and unmarshalling, to the client stub.
The client application receives the unmarshalled response as a return value.
This will look like it was a local method call.

Well established pattern.
Frameworks, implementation details and performance are the only real changes.
API developers need to pick the right framework, define the API as well as the relevant types using a RPC framework.
The description then needs to be published.

The client and server in the system are decoupled.

When the system has been designed and implemented a stub can be generated for new clients.
New clients just need to generate their client stub to use the server method based on the published API definition.

Using RPC, client and server is not restricted to programming language.

\subsection{RPC: Pros}
CConvenient - generate stubs which look just like local methods on objects.
Communication details, and data passing details are abstracted away from the developers.
Any failures result in just exception or error depending on the programming language just like normal methods.

\subsection{RPC: Cons}
RPC methods are less reliable and are slower.
This can lead to performance issues.
Client will not know how long Procedure calls will take.

\subsection{Considerations}
API design will need to consider this.
A solution is to use asynchronous versions for slow methods which is a best practice.
Another consideration is idempotency in the case that a message is lost or slow.

This is generally used when backend systems are communicated to one another but frontend systems can be considered.
RPC is a good style when providing an API to a different company vs an end user or a webpage.
Good for different internal components within a large scale system.

The RPC API style is also a good way to abstract away the network communication and focus on the actions the client wants to perform on the server.

This would be a bad fit if we want to use browser cookies or headers.
There are other styles for this.
RPC revolved more around actions rather than data and resources (Think CRUD applications).

\begin{itemize}
    \item gRPC
    \item Apache Thrift
    \item Java Remote Method Invocation (RMI)
\end{itemize}


\section{Representational State Transfer - REST}
Set of architectural constraints and best practice for web APIs. It is just a style allowing / making it easier to achieve / improve quality attributes.
A RESTful API meets a certain traits.
More resource orientated.
Encapsulates the different entities in a system and allows for the manipulation of those resources through only a small number of methods.
In a REST api, a client requests a named resource and the server responds with the current state of that resource.
Usually done using HTTP. The resource can be implemented in a completely different way, only a representation of the resource is sent.
Resource is just an abstraction.
Interface is more dynamic, actions are not statically defined like in RPC.
This is through a concept called HATEOAS - hypermedia as the engine of the application state.
This is achieved by accompanying a state representation reasoner to the client with hyper media links.
The client can follow these links and progress its internal state.

\subsection{Achieving quality attributes}
Important requirement of a system that provides RESTful API is that the server is stateless and does not maintain any session information about the client.
Each message should be served by the server in isolation without any information about previous requests.
This allows for high availability and scalability.
If there is no session information high groups of servers can be used and the load can be spread. Client will no notice.
Cache-ability - Important requirement.
This means the server either implicitly or explicitly defines each response as cacheable or non-cacheable.
Allows the client to eliminate the potential round trip to the server and back if they response is cached somewhere closer to the client.
This will also reduce system load.

\subsection{Resources in RESTful APIs}
Each resource is named and addressed using a URI.
Resources are organised in a hierarchy where each resource is either a simple resource or a collection resource.
The hierarchy is represented using forward slashes.
A simple resource has a state and optionally can contain sub resources.
A collection resource is a special resource that contains a list of resources of the same type.

\subsubsection{Naming resources - best practices}
Use nouns only.
Provides distinction between actions and the resources the actions will be taken on.
Make a distinction between collection resources, and simple resources using plural and singular.
Use clear and meaningful names.
This will make the API easier to use for developers, and avoid mistakes and incorrect usages.
Final practice.
Use Unique, URL friendly names for resource identifiers for usage via the web.

\subsection{Methods and operations}
REST API limits the number of methods we can perform on each resource to a few predefined options.
Creating, Updating, Deleting and getting the current state.
When the resource is a collection resource, getting its state usually means getting a list of its sub-resources.

Since REST APIs are commonly implemented using HTTP the operations are mapped to HTTP methods as follows.

\begin{itemize}
    \item POST - Create an existing resource.
    \item PUT - Update an existing resource.
    \item DELETE - Delete an existing resource.
    \item GET - Get the state of an existing resource.
\end{itemize}

Additional can be defined but the situation is uncommon.

\subsection{HTTP Semantics}
Guarantees from HTTP semantics.
GET is considered safe i.e it will not change the state of the resource.
GET, PUT, DELETE are idempotent.
i.e same result if you apply those operations multiple times.
GETs are considered cacheable by default while responses to POST requests can be made cacheable by setting the appropriate HTTP headers sent as part of the response to the client.
This feature allows us to conform to the cachabilty requirements of a REST API.
When the client needs to send additional information to the system via a POST or PUT command JSON can be used.
XML is also usable.

\subsection{Step to creating a RESTful API}
\begin{itemize}
    \item - Identify entities in a system to serve as resources
    \item - Mapping the entities to URIs and organising the hierarchy of resources based on their relationships
    \item - Choose a representation for each resource. Can use JSON for this and include links for HATEOS
    \item - Final step - assign HTTP methods to actions.
\end{itemize}


\section{REST API Design in Depth}
%TODO combine notes

\subsection{Setting the stage}
Apps has bad reviews and is dying as there are no active users.
Management has become concerned.
This decline has occurred because competition has better technologies that make their offering more attracted to consumers.
Competitors are also stealing partners with affiliation deals.
Management has now realised that they must invest in digital technologies or continue to lose market share.
API is core to this businesses survival strategy.
Over time the company will gain popularity and system will expand.

\subsection{Evolution of RESTful Services}
API - user interface to data and systems that is consumed by applications rather than humans.
A well define contract between provider and consumer
\textit{Contract = Structure of Request and Response XML}

\subsubsection{JSON vs XML}
XML is heavy in terms of network traffic
Parsing of XML is CPU is Memory intensive
Slow performance of XMl on browser front end is not desired
Mobile devices batter performance reduced due to XML
Standards \& versions of standards caused confusion and complexity

XML-RPC and SOAP are API standards that use XML
With the rise of mobile and single page apps REST/JSON have gained popularity
APIs are leveraged across many industries

\subsubsection{REST}
Architectural style and Set of Principles not Technology specific or standard specific

Resources are objects, things and entities
Resources have representational state managed in backend
REST is NOT a standard but an architectural style
REST may use the HTTP application protocol

\subsection{REST API Architectural Constraints}
Six constraints for a API to be RESTful
\begin{itemize}
    \item 1) Client - Server
    \item 2) Uniform Interface
    \item 3) Statelessness
    \item 4) Caching
    \item 5) Layered System
    \item 6) Code on demand (Optional)
\end{itemize}

\subsubsection{How to know if an architecture is RESTful?}
Richardson Maturity model
\begin{itemize}
    \item Level 0: The Swamp of POX
    \item Level 1: Resources
    \item Level 2: HTTP Verbs
    \item Level 3: Hypermedia Controls
\end{itemize}

\subsubsection{REST Architectural Constraints - 1 - Uniform Interface}
Client and Server share a common technical interface
Definition of an interface
\begin{itemize}
    \item Contract for communication between client-server
    \item Contract has NO business context
    \item Contract defined using HTTP methods and media types
\end{itemize}
Uniform Interface De-Couples the architecture; components can evolve independently

There are four guiding principles
- 1) Identity of the resource (URI / URL)
- 2) Representation of the resource
- 3) Self descriptive messages - metadata
- 4) Hypermedia

\subsubsection{Client Server - 2}
Client and server do not reside in the same process
- RPC over HTTP
- Server Decoupled from CLIENT

\subsubsection{Statelessness - 3}
Each client request is independent
Server receives all info it needs in the request

\subsubsection{Caching - 4}
Statelessness - Negative impact on performance
Chattiness
Higher data traffic (impact's scalability

Caching - Performance
- Scalability
- Reduce Chattiness

Caching counter-balances some of the negative impacts of Statelessness

\subsubsection{Layered System - 5}
- Client - Server architecture consist of multiple layers
--- No one layer can see past the next
--- Layers may be added, removed or modified based on needs

\subsubsection{Code on demand (Optional)}
Server can extend client's functionality by sending the code

\subsection{Designing REST API}
Design your API with the needs of the App Developers in mind
Use nouns to identify resources, use of plural is suggested
Actions
--- API Operation $ \neq $ CRUD

OK to use action as part of resource hierarchy or create an action with subquery

1) Create, a simple base URL; subdomain; separate domain
2) Resources names should be nouns; Use of plurals is fine
3) Actions can be verbs, may be standalone
4) URI Pattern for associations; avoid nesting over three levels

\subsubsection{CRUD operations}
POST - Creates a resource,
201 is a success, may return a link (id) to new resource in Location header
- May return the new object

Failure - 4xx Bad Request ---- 400 missing field etc
5xx Issue in processing - e.g 503 database unreachable

GET - reads/ retrieves a resource collection or specific resource
200 - Send back the response in reqeusted format
Failure - 4xx Bad Request - e.g 404 Resource no found
5xx - Issue in processing - e.g 500 Internal Server Error

PUT/PATCH Request Handling - Updating a resource
- PUT - Updates all attributes of existing resource - effectively replace. Can also CREATE if client provide ID (not suggested)
- PATCH - Modifies parts of an existing resource

Success - 200 - Resource in body of response - OPTIONAL
204 - No Content
201 - Created - No need to send the Link - OPTIONAL

Failure - 4xx Bad Request - e.g 404 Resource Not Found
5xx - 503 Database unreachable

DELETE - Deleting a resource
Success 200 - May return deleted resource in the response body
204 - No returned content = 204
Failure - 4xx Bad Request - 404 Resource Not Found
5xx Issue in processing - 503 database unreachable

Rougly speaking restrict the number of HTTP codes to a maximum of 8
---> 200/400/500 \textbf{MUST} be used

\subsubsection{An alternative approaches}
Always send back HTTP Status = 200 OK
--- Place details in the payload

\subsubsection{API Changes}
Non - Breaking
- Adding a new operation or resource
- Adding optional parameters or properties

Breaking
- Change the HTTP verb or methods
- Delete an operation

\paragraph{Handling Change}
- Avoid Changes : Is the change really adding value?
\begin{itemize}
    \item Eliminate or minimises impact on app developers
    \item Provider planning opportunity to the app developers
    \item Support backward compatibility (if possible)
    \item Provide support to app developers with the changes
    \item Minimise the change frequency e.g (once per 6 months)
\end{itemize}
\textbf{Version your API right from day 1}

\subsubsection{Version information}
%TODO add code listings
HTTP Header
Query Parameter
URL

\subsubsection{Multiple Version Support Key points}
\begin{itemize}
    \item Support at least 1 previous version for a period of time ---- 3 months/ a year etc
    \item Mark the previous version as deprecated. \textit{New app developers can only access the latest version}
    \item Publish a roll out plan in advance
    \item Manage the change log that clearly shows the reason for the new version
\end{itemize}

\subsubsection{Design Decisions}
\begin{itemize}
    \item Which component should control the caching
    \item What to cache?
    \item Who can cache?
    \item For how long is the cached data valid?
\end{itemize}

\paragraph{Why cache?}
Benefits - Enhances performance, Leads to higher scalability

Data to cache depends on - Speed of change, Time Sensitivity, Security

Design decisions - Who can cache?, for how long?

\subsubsection{HTTP Cache control}
Responses
WHo can cache the response?
FOr how long?
Under what conditions?

Request
- Override the caching behaviour
- Protect sensitive data from caching

\paragraph{Practices}
Take advantage of caching especially for high volume APIs
Consider no-store and private for sensitive data
Provide the validation tag (ETAG) especially for large responses
Carefully decide on the optimal max-age

\subsubsection{Partial Response}
Better performance and optimised resource usage - CPU, memory, bandwidth
API Consumer controls the granularity
Common API version for all consumers
---- E.g. to support multiple devices, use cases - form factors

\subsubsection{Pagination}
Cursor - "Control structure that enables traversal of records"
Cursor based pagination considered most efficient

Offset-based pagination - Commonly used approach %TODO add example /thing?offset=6&limit=5

HTTP Link Header - Use of a Link header a.k.a web linking

\paragraph{practices}
- Consider pagination for avoiding large data sets
---- Decide on the default page size may be different for API(s)
- Support for multiple pagination for different resources

\subsection{REST API Error Handling Patterns}

\subsection{REST API Handling Change - Versioning Patterns}

\subsection{REST API Cache Control Patterns}

\subsection{REST API Response Data Handling Patterns}

\subsection{REST API Security}

Basic authentication ---- not a good idea with HTTP (sending encoded credentials over the network in Authorisation header in HTTP reqeust). Better to use with TLS/ HTTPS

\subsection{REST API Specifications using Swagger 2.0 / OAI}

\subsubsection{Contract creation approach}
Contract Last - API developer -> code and then specs
Contract First - API Developer -> Specs and then code

\textbf{Adopt the Contract First approach for REST API design and development}

Swagger is a standard to describe RESTful API
Why?
Developer documentation
Automated Proxy Creation, Server code generation
Client code generation
Service Virtualization (Mocking)

Defacto standard supported by multiple vendors

- Multiple path definitions
- Each path can have multiple operations

Can be edited with any YAML tool

\subsection{API Management}

API Management Scope
Lifecycle - Productivity
Security
Traffic - Analytics
Productise - Monetization


Lifecycle
- Establish processes, practices and roles
- Setup App developer communication
- Publish roadmap in advance
- Leverage and create tools

\subsubsection{Productivity}
App Developer
- API Documentation
- Self serve provisioning
- Support

API Developer
- Development guidelines
- Frameworks
- Best practices

\subsubsection{API Security}
Leverage API management platforms for security
Authentication and Authorisation - OAuth Server, Key/Secret Management
Functional Attacks - Protection from known and future threats

\subsubsection{API Security practices}
Educate the API developers
Provide development guidelines on security
Stay up to date on security threats
Have a well defined security policy
Develop a contingency plan
Work with the other stakeholders to infrastructure

\subsubsection{API Traffic Management}
- Response time consistency
- Service Level Agreement
- Protecting the backend

\paragraph{Traffic management}
Quota - Defines the maximum number of calls per unit time
Rate Limiting - Lists number of concurrent connections to API
Spike Arrest - Prevents calls beyond high water mark from reaching the backend %TODO see graph

\paragraph{Analytics}
Performance
- Response Time
- Throughput
- Peaks/Valleys

Errors
- API Errors
- Backend errors

SLA
- Calls / time
- Response

Usage?
%TODO go through this in a bit more depth

Transactions
- Specific to the implementation
- Logic built into the proxy

\subsubsection{API Management: Product and Monetise}
API = Product
Business base ---- New revenue, brand awareness, partnership
Research - Consumer, need and wants, why use the API
Create - Delivery/planning and pilot
Marketing - Events, Promote
Launch Post Launch - Launch/ Support

\textbf{You need to tread an API like a product}

\subsubsection{Monetization}
%TODO add a diagram describing

\paragraph{Technology considerations}
Tiered product definitions
Usage metering/ subscription management
Reporting / partner management


\section{GraphQL}

%
\section{GraphQL}

Dedicate this part to graphQL

