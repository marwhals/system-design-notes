\chapter{Data storage at Scale}

Motivations: Choosing the right databases amongst many options.

\section{Relational Databases and ACID Transactions}

\subsection{Relational Databases}
Data is stored in tables.
Each row in a table corresponds to a single record and all the records are related to each other through predefined columns they all have.
Each column in a table has a name, a type and optionally a set of constraints .... NULL etc.
The relationship between all records inside a table is what gives this type of database the name relational database.
Structure of each table is decided ahead of time and is referred to as the schema of the table.
We know what each record in the schema must have because it is predefined.
We can use a very robust language to query the data (analyse and update) in the table.
This is SQL.
Different implementation add their own features to this language.
Oracle etc....
Majority of operations are the same across all relational databases.
Proven way of storing structured data.
Avoids data duplication when memory is usage has to be considered (large scale)
Use joins to combine information from multiple tables without duplication.

\subsubsection{Relational Databases - Advantages}
Carryout complex and flexible queries using SQL. Use for analysis
Save memory because multiple tables can be joined ..... save costs.
Easy to reason about, very natural for humans.
No sophisticated knowledge required.
Most importantly, provides ACID transactions.

\begin{itemize}
    \item Atomicity
    \item Consistency
    \item Isolation
    \item Durability
\end{itemize}

In the context of transactions the sequence of operations should look like a single operation externally.
Relational databases guarantee atomicity of transactions.
Atomicity - transactions appear once or not at all.
Consistency - guarantees that transaction that has been committed will be seen by all future queries and transactions.
Also guarantees that data constraints that are set for the data, are met.
Isolations - relates to concurrency, if two transactions are taking place the second transaction will not see an intermediate state.
They are separate.
Durability - once a transaction is complete its final state will persist.

\subsection{Relational Databases - Disadvantages}
Rigid structure enforced by database schema.
Schemas have to be designed ahead of time before the table can be used.
Future changes to the schemas can lead to downtime.
Changes ideally should be avoided and not done at all.
Through planing is required for this.
Harder and more costly to scale because of their complexity.
Providing SQL and ACID transactions is not straight forward and thus relational databases are harder to maintain.
Due to ACID guarantees, reads are slower to perform compare to other types of databases.
Different implementations have different performance optimisations and guarantees.

\begin{note}
    Generally, relational databases are slower than non-relational ones.
\end{note}

When choosing one consider pros/cons and the use case.
I.e. is the data related, are reads important?.


\section{Non-Relational Databases}

NoSQL databases \textit{generally} allow for the logical grouping or records without them having to conform to some kind of schema.
Can easily add some kind of attributes to the different records without redesign or effecting other records.

\textit{Most} languages do not have tables as a structure, external libraries required.
NoSQL uses more typical computer science data structure (arrays, trees etc).
This eliminates the need for Object Relational Mappings ORM (Hibernate) to translate business logic for storage in a database.

Relational Databases are designed for efficient storage.
That is they optimise for low memory availability and expensive memory.
NoSQL databases on the other hand are typically optimised towards faster queries.
Different types of non-relational databases are optimised for different types of queries based on the use-case.

\subsection{Some issues with flexible schemas}
The loss of ability to easily analyse those records since each record can have different structure and data.
Joining becomes very hard, these operations are often not supported by NoSQL databases, or are hard.
Each NoSQL database supports a different set of operations and different set of datastructures.
ACID transactions guarantees are rarely supported by non-relational databases (but there are exceptions).

\paragraph{Types of NoSQL Databases}
Three types of NoSQL databases.
The categories, however are somewhat blurry.

\subparagraph{First Type - Simple Key/Value store}
We have a key that uniquely identifies a record and value that represents the data.
The value can be anything, primitive or a binary blob.
Can be though of as a large scale hash table/dictionary with few constraints on the type of value that can be held for each key.
Good for caching pages or for quick fetching and easy querying.

\subparagraph{Second Type - A document store}
Collections are stored as documents.
Documents have a bit more structure.
Each document can be thought of as an object with different attributes.
Those attributes can be different types.
Similar to classes and fields.
Documents are easily mapped to objects inside a programming language.
Think JSON, YAML, XML

\subparagraph{Third Type - A graph database}
An extension of a document store but with additional capabilities to traverse, link and analyse records more efficiently.
These types of databases are particularly optimised for navigating and analysing relationships between records in a database.

\paragraph{Use Cases}
Fraud detection.
Recommendation engines.

\subsection{When should they be used?}
Look at use case and analyse what is required from the database and what can be compromised.
Non-relational databases are better when it comes to query speed.
They are also good for caching.
They can store common query results that correspond to user views or pages.
In memory "key,value" stores are optimised for this.
Real time big data is a good use case, since relational databases are too slow and not scalable enough.
Another use case is when the data is not structured and different records can contain different attributes.

\subsection{Key-Value stores}
\begin{itemize}
    \item Redis
    \item Aerospike
    \item Amazon DynamoDB
\end{itemize}

\subsection{Document Stores}
\begin{itemize}
    \item Cassandra
    \item MongoDB
\end{itemize}

\subsection{Graph Databases}
\begin{itemize}
    \item Amazon Neptune
    \item NEO4j
\end{itemize}


\section{Techniques to improve performance, availability and scalbilty of databases}
There are three techniques to improve the scalability, performance and availability of a database.

\subsection{Technique one - Indexing}
Speeds up retrieval operations and locate the results in a sublinear time.
Without indexing those operations could require a full table scan, which is bad for large tables.
Example operations
Sorting, search.
If performed often this can lead to a performance bottleneck.

An index is a helper that we create from particular column or group of columns.
When the index is created from a single column, the index table contains a mapping from the column value to the record that contains the value.
Once that index table is created, we can put that table inside a data structure (good structures - B-Trees(or any self-balancing tree), or a Hashmap).
This will keep the values sorted and thus makes searching more efficient (think big O complexity).
Searching is log n.
Sorting is n log n.

This will keep the values sorted and thus makes searching more efficient (think big‑$O$ complexity).
Searching is $O(\log n)$.
Sorting is $O(n \log n)$.

Composite indexes can be created.

%TODO add reminder on normal form........for relational databases

\subsubsection{Tradeoffs}
Optimising for one operation can lead to a tradeoff elsewhere.
Indexing increases memory and decreases the speed of writes but increases the speed of reads.
Writing records becomes slower because each time a write, or new record is added the index table must also be updated.
Indexing is also used in non-relational databases to speed up queries.

\subsection{Database replication - single database, single point of failure.}
Solution, replicate data and run multiple instances of the database on different computers.
This increases fault tolerance, availability.
One replica goes down another can take its place and business is not effected.
Queries can continue going to available replicas while the faulty replica is fixed.

Get better performance and better throughput.
Better throughput by distributing our queries across multiple computers/ replicas.

\subsubsection{Tradeoffs}
Introduces higher complexity especially in regard to write, update and delete operations.
Making sure concurrent modifications to the same records don't conflict with each and providing predictable guarantees in terms of consistency and correctness is not a trivial task.
Distributed databases are very hard to design, configure and manage, especially at scale.
Requires knowledge of distributed systems.

Database replication is widely supported by most modern databases
Non-relation databases support replication (high availability, large scale) out of the box as they were designed with that in mind.

Support for replication with relational databases varies amongst different implementations.

\subsection{Database partitioning (Sharding)}
Split data amongst different database instances for better performance.
Each instance will run on a different computer typically.
More computers, more data.
Queries which use different partitions can be done in parallel.
We get both better performance and better availability.
Sharing effectively turns the database into a distributed database.
Add complexity and overhead since routing is required for the right shard/partition as well as making sure one shard does not become too large.

Database sharding is a common feature in most non-relational databases.
Since by design they decouple different records from each other.
Storing records on different computers is a lot more natural and easier to implement.

For relational databases, it depends on the implementation.
Relation database queries usually involve more than one record / are more common.
Splitting these across multiple machines is more challenging to implement in a performant way while supporting things like ACID transactions or table joins.
When choosing a relational database for high volume of data use case check for partitioning support.
Partitioning can also be used to split infrastructure like compute instances and redirect traffic to different machines.
Can also do this based on mobile/browser etc running the same application.

This routing allows for understanding which users are affected during an outage.

\begin{note}
    These three techniques are orthogonal to each other.
    They are often used together in real user use cases.
\end{note}


\section{Brewers CAP Theorem}
%TODO find a good diagram for this

\begin{note}
    In the presence of a network partition, a distributed database cannot guarantee both consistency and availability.
\end{note}

The scenario of database instances not being able to communicate with one another.
When an isolated server cannot communicate with its replicas it can either return an error or produce inconsistent data.

\begin{note}
    No network partition then no tradeoff and both consistency and availability can be offered.
\end{note}

\begin{itemize}
    \item Consistency - Every read request received the most recent write or an error regardless of which instance is serving when considering a network partition.
    \item Availability - Every request receives a non error response without the guarantee that it contains the most recent write.
    Occasionally different clients may get a different version of a record, some may be stale but all requests return a successful value.
    \item Partition tolerance - means that the system continues to operate despite an arbitrary number of messages being delayed or lost by they network.
\end{itemize}

The theorem basically states one of these must be dropped when we are working with a database that may have to deal with a database partition.
Could have a single database to guarantee consistence and availability. I.e no network involved with the database, but this would not scale.
When architecting for partition tolerance we either have to drop availability or consistency.
Depends on the use case.

The choice is not so distinct, more of a spectrum.
More availability less consistency, less consistency more availability.
Depends on the tolerance requirements of the application.

Important to get these tradeoffs in the architectural design phase, when we formalise non-functional requirements.


\section{Scalable unstructured data storage}
%TODO where does pine cone and vector data bases fit in...........
Unstructured data, data that doesn't follow a model or schema.
Non relational data still has keys and values and structure.
Consider binary files.
Databases may allow for storage but they are not optimised for it.
Limit on the object size with DBs otherwise we would encounter scalability and performance problems.
Use cases
\begin{itemize}
    \item Disaster recovery
    \item Archiving
    \item Web hosting media like videos etc
    \item Collecting data points for machine learning purposes
\end{itemize}

Huge datasets, storage solution needs to be scalable.
Each object can also be big.

\subsection{First solution - Distributed file system - network of storge devices}
We can get different replication consistency, and auto healing guarantees based on the file system.
Main feature is that files are stored in a tree like structure.
Benefit of storing unstructured data like this is that special APIs are not needed.
Can easily modify these files.
Performance intensive operations such as big data analysis or transformations operations on data are very fast if we do it on a distributed file system.
\begin{note}
    Very useful for machine learning projects.
\end{note}

\subsubsection{Limitations of Distributed file systems}
\begin{itemize}
    \item limited in the number of files we can create which is scalability issue with relatively small files like images.
    \item easy access to those files via web api is hard, Additional abstractions would be required.
\end{itemize}

\subsection{Another solution - use an object store}
\begin{itemize}
    \item A scalable storage solution designed for unstructured data at an internet scale.
    \item Can scale linearly like a distributed file system by adding more storage devices.
    \item Unlike a distributed file system we have virtually no limitation on the number of binary objects we can store in it
    \item High sizes on the size of a single object....can be terabytes.
    \item Good for archiving and backups
    \item Features of typical object stores can include HTTP rest API that makes them effective for storing multimedia content - images etc that can be linked to a webpage
    \item Object versioning is another feature, on a file system we would need to use an external versioning system.
\end{itemize}

\subsection{Differences between object store and a distributed file system}
\begin{itemize}
    \item No directory hierarchy
    \item Flat structure called buckets
    \item main abstraction is an object
    \item Typically key value pairs, with value being the content. Additional key value pairs exist for meta data like file type and size.
    \item Objects can include an access control list who can access who has read/write permissions.
\end{itemize}

Commonly provided by cloud providers.
Typically broken into classes with each class providing different prices and SLA guarantees.
Top tier is usually availability of a high percentage, offers lowest latency and highest throughput
Top tier is best for data that will be frequently accessed lke content videos, images etc.
Middle tiers have lower guarantees for viability and are cheaper.
- These options have limited performance and sometimes have limits frequency of usage as well.
- Good for data backups, i.e data not needed often
Lowest tier is good for longer term archiving.
Usually cheap and used for specific use cases.

\subsection{Cloud vs OpenSource vs Third Party Managed}
Cloud may not be an option for budget, legal or performance constraints.
There are options for on premise storage devices.
Some follow the same APIs as some cloud vendors, so they can easily be used in hybrid cloud environments (cloud + private data centre).
Can use the same object ID to store data in both locations.

Just like distributed file systems, object stores use data replication although this is abstracted away.
Ensure physical storage loss will not result in loss of actual data.

\subsection{Object store downsides}
Data is immutable, no edits can only be replaced with a newer version.
Performance implications: storing large documents that require amends would not be feasible
Needs a special API or a REST API unlike a distributed file system.

Distributed systems are the preferred storage solution over object stores for high throughput solutions.

\subsection{Scalable Unstructured Data Storage Solutions}
Cloud based
\begin{itemize}%TODO look into differences
    \item Amazon S3
    \item GCP
    \item Azure
    \item Alibaba
\end{itemize}

Open source and third party
\begin{itemize}
    \item OpenIO - software defined
    \item MinIO - S3 and native to kubernetes
    \item Ceph - \textbf{Open-source} reliable and scalable.
\end{itemize}