\chapter{Big Data Architecture}

\section{Big Data meaning}
Data sets which are too big or too complex or are produced at too fast of a rate such they that exceed the capacity of a traditional application.
Characteristics are typically volume, variety, rate/velocity.
This data can be used to gain insights/conclusions via visualisation, querying or predictive analysis.
Can also be used to find anomalies, analysing logs.

\section{Batch}
Requires the storage of incoming data in a distributed database or a distributed files system.
Data is never modified, it is only added to the end.

\subsection{Batch - Key Principle}

The Key principle here, is that the data is processed in batches or records on a fixed schedule, or a fixed number of records that we want to process.
The schedule can be adapted for the users/clients needs.
Each time a batch processing job runs in can process new data and produce and up-to-date view of all the current data.
This can be stored in a well-structured and indexed database that can be queried to get insights.
Could analyse recent data or the whole data set.
This data is not processed in real time.

\begin{note}
    This data can be used to create a Machine learning model.
\end{note}

\subsection{Batch - Pros}
Batch processing provides the user with high availability, no downtime for users and the old data view is still available.
Batch processing is more efficient versus processing each piece of data individually.
There also a higher tolerance towards human error, with regard to bad code / deployment issues.
Batch processing can be used to perform complex data analysis of large data sets.

\subsection{Batch - Cons}
Long delay between data coming in and the result we get from the processing job.
The view is not real time which can be an issue in some use cases (Trading, 24/7 systems etc).
This forces users to wait a long time before they can act on the insights from the system.
They may not know that the data is not in real time.

\section{Streaming - Real time processing}
Each new event into the system is placed into a queue or a message broker, on the other end there is a processing job that processes each bit of data as it comes through.
After processing, the processing job updates the database that provides querying capabilities for real time visualisation and analysis.

\subsection{Streaming - Pros}
Can respond to data immediately

\subsection{Streaming - Cons}
Hard to do any complex analysis in real time and as a result insight may be poor compared to batch.
Hard to data fusion in real time, at different time points or analysing historic data.
Only limited to recent data for predictions.

\section{Lambda architecture}

May need the approaches/processes of both strategies.
Lambda architecture takes advantages of both.
Aims to find the balance between fault tolerance and comprehensive analysis of the data from batch processing and the low latency that we get from real time.
In this architecture the infrastructure is divided into three layers:

\begin{itemize}
    \item Batch
    \item Speed
    \item Serving Layer
\end{itemize}

\begin{note}
    Data that enters the system is dispatched into both the batch layer and the speed layer simultaneously.
\end{note}

\subsection{Batch Layer}

The purpose of the batch layer is to manage our data set and be the system of records.
The data in our master set is immutable and new data is appended, never modified.
This data is usually on a distributed files system, optimized for storing big files containing massive amounts of data.
The second purpose of the batch layer is to pre computer our batch views.
Every time we run our batch processing job, it processes all the data that we have in out master data set.
Once the processing is complete it indexes and store the data in a read only database.
Typically, this overrides the existing pre-computed views that we created the previous time we ran the processing job.

\begin{note}
    The batch layer aims at perfect accuracy and operates on the entire data set.
\end{note}

\subsection{Speed Layer}

Data is sent to this layer in parallel.
Real time strategy is used here.
All the data goes into a queue or a message broker and is then picked up as it arrives by the processing job.
Processing job analyses the new even and adds the processed even to the real time view ready for querying.
This layer compensates for the latency in the batch layer.
Unlike the batch layer it only operates on the most recent data and doesn't attempt to provide a complete view or make any data corrections.

This layer exists between the last time a batch was run and the present moment/recent data.

\subsection{Serving layer}
The serving layers purpose is to respond to queries and merge the data from both the batch and speed layer and update the real time views.

%TODO consdier kappa